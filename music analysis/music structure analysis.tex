\chapter{A little bit of theory on Music Structure Analysis (MSA)}

\section{Fundamental structure}

In Schenkerian analysis, the fundamental structure, also known as Ursatz, represents the underlying structure of a tonal work at the most remote or background level, in the most abstract form. It comprises an essential elaboration of the tonic triad, including the entire line and accompanying bass arpeggiation. The fundamental structure, like the fundamental line, takes one of three forms, depending on which pitch of the tonic triad serves as the primary tone. For example, in the key of C major, the fundamental structure would be represented by the fundamental descending line from scale degree 3.

The Urlinie, or fundamental line, unfolds the basic triad and presents tonality on horizontal paths. The tonal system also flows into this to bring a sense of purposeful order to chords by selecting harmonic degrees. Voice leading serves as the mediator between the horizontal formulation of tonality presented by the Urlinie and the vertical formulation presented by the harmonic degrees.

The upper voice of the fundamental structure, represented by the whole line, utilizes a descending direction. In contrast, the lower voice, the bass arpeggiation through the fifth, takes an ascending direction (as shown in figure 1). The combination of the fundamental line and bass arpeggiation constitutes a unity, and neither can exist alone; it is only through their unification in a contrapuntal structure that they produce art.

\section{Structural levels}

In Schenkerian analysis, structural levels refer to the representation of a piece of music at different levels of abstraction. These levels typically include the foreground, middle ground, and background. Schenker posits that musical form, as represented through these structural levels, can be understood as an energy transformation, specifically the transformation of forces that flow from the background to the foreground through the various levels.

\section{Meter}

In music, meter refers to the recurring patterns of accentuation and the organization of those accents into regular units, such as bars and beats. These patterns are implied by the performer and expected by the listener, even if they are not necessarily sounded.

Different cultures have developed various systems for organizing and performing metrical music, such as the Indian tala system and similar systems in Arabic and African music. Western music inherited meter from poetry, where it denotes the number of lines in a verse, the number of syllables in each line, and the arrangement of those syllables as long or short and accented or unaccented. The first systematic method of notating rhythm in Western music was derived from rhythmic modes, based on the basic types of metrical units in the quantitative meter of classical ancient Greek and Latin poetry.

\section{Music form}

Music form refers to the structural organization of a musical composition or performance. This can include elements such as the arrangement of musical units (such as rhythm, melody, and harmony) that exhibit repetition or variation, the arrangement of instruments, or the orchestration of a symphonic piece. These elements interact to shape the composition and create a meaningful musical experience for the listener. The organizational elements can be further broken down into smaller units called phrases, which express a musical idea but lack sufficient weight to stand alone. The form of a composition unfolds over time through the expansion and development of these ideas. In tonal harmony, form is primarily articulated through cadences, phrases, and periods. 