\chapter{About Musical Analysis}

\section{Heinrich Schenker}

Heinrich Schenker was a prominent figure in music theory, known for developing the analytical method called "Schenkerian Analysis." This method is primarily applied to the study of tonal music. It is based on the concept of "the Ursatz," or fundamental structure, which consists of three levels: foreground, middle-ground, and background.

The foreground level is the surface structure of the musical composition, comprising the melody and harmony. The middle-ground level is the next structural level beneath the surface, while the background level represents the underlying tonal structure of the piece.

Schenker's approach to analysis postulates that all tonal music can be understood as a series of "preparations" and "resolutions" of tonal tensions. This approach emphasizes the horizontal dimension of music, specifically melody, and the relationship between individual notes and the piece's overall structure, as opposed to the vertical dimension of harmony.

Schenker's ideas have significantly impacted the field of music theory and analysis, and his method is still widely applied in the study of tonal music. Schenkerian Analysis is employed in musicology, composition, and music education and is considered a valuable tool for comprehending the structure and meaning of musical compositions.

Schenkerian Analysis can be considered a type of multi-level analysis, as it breaks down a piece of music into different levels of structure, with each level representing a different level of complexity and abstraction.

This hierarchical approach to analysis, where different levels of structure are examined, is a characteristic of multi-level analysis in general. Therefore, Schenkerian Analysis can be considered as a form of multi-level analysis, as it analyses the piece of music on multiple levels of structure and abstraction.

While Schenkerian Analysis is primarily a theoretical approach to music analysis that relies on human interpretation and notation, it is possible to use computer algorithms, such as convolutional neural networks (CNNs), to analyze music similarly.

It is important to note that even though it is possible to use CNNs for music analysis, Schenkerian Analysis is a highly theoretical and interpretive approach that relies on human expertise and interpretation. It's hard to replicate the same results using a machine learning model like CNNs.

\section{George Russell}

"The Lydian Chromatic Concept of Tonal Organization" is a theoretical treatise by George Russell, a jazz musician, and composer. First published in 1953, the book presents a system of tonal organization based on the Lydian mode. Russell asserts that the Lydian mode, characterized by its raised fourth scale degree, is the most natural and fundamental tonal system and that all other modes and scales can be derived from it. Furthermore, he claims that the Lydian mode is the only tonal system to organize the various pitch collections utilized in contemporary music appropriately.

The book provides a methodology for arranging chords and scales based on their relationship to the Lydian mode and includes a technique for improvisation and composition utilizing this system. The book's purpose is to present a new approach to music theory that would benefit jazz musicians and composers, as well as other musicians.

The book is considered a seminal work in jazz education and has significantly impacted jazz harmony and education development. Many jazz musicians and educators have studied and applied Russell's concepts in their work.

In summary, "The Lydian Chromatic Concept of Tonal Organization" is a theoretical work that posits the Lydian mode as the most fundamental and natural tonal system and organizes chords, scales, and improvisation based on this mode.

\section{Leonard Bernstein}

What we're trying for is a very high overview of musical development in terms of a vocabulary constantly being enriched by more and more remote and chromatic overtones. It's as if we could see the whole of music developing from prehistory to the present in two minutes. Let's again pretend we're hominids, and that the smash hit of the moment is, let's say, Fear Harvard. Here we are in our hominid hut, crooning, Fear Harvard relations to age, and now maybe our wives and maybe our prepubescent sons join in, and automatically we're singing not in unison, but in octaves, since men's and women's voices are naturally an octave apart. Now that octave interval, I wish I could sing an octave so I could really show you what I mean, but that octave interval happens to be the first interval of the harmonic series as you remember, right? Okay, now centuries pass, and the next interval of the harmonic series is assimilated by humanity. Namely, the fifth. And now we can be singing this, by those festival rides on the horizon. Now, of course, this little change brings us forward a mere 10 million years into the 10th century AD and into a fairly sophisticated musical culture. But now we admit the next interval of the series, the fourth, and now we can mix intervals of the octave and the fifth and the fourth. That's beginning to sound like polyphony. And again comes a great leap as music absorbs the next overtone, the third, and just listen to the difference. It's a whole new music, richer, mellower, with a new coloristic warmth, and as we know, this new interval of the third, because I like the older sound better, but anyway, as we know, this new interval of the third introduces into music the phenomenon of the triad so that now, fair Harvard, can begin to sound more like its Victorian self. And so there is born what we now call tonal music, a stable tonal language firmly rooted in the basic notes of the harmonic series, the fundamental and its first different overtone, the fifth, now and forever more to be known as the tonic and the dominant. And that fifth interval really does dominate, because once this tonic-dominant relationship is established, it's a field day for composers. There can now be fifths of fifths of fifths of fifths, each one of them a new tonic producing a new dominant, a whole circle of fifths, 12 of them in fact, always winding up with the starting tone, where the preceding upwards, let's say from the low C, G, D, A, E, B, F, sharp, C sharp, G sharp, D sharp, B flat, F, C, that's again C, believe it or not. Or preceding downwards, starting from that, C, F, B flat, E flat, E flat, G flat, G flat, C, back to C. That's a circle of 12 fifths, and that's the answer I promised you. That's how we get the 12 different tones of our chromatic scale. In other words, if you take all those 12 of the circle of fifths and put them together in scale order, you'll get this. And what's more, those 12 notes generate a circle of 12 keys, through which, thanks to the perfecting of the temperate system, composers can now go freewheeling at their own chromatic pleasure. Now this means that ultimately, fair Harvard can sound like this. Now that's chromatic porridge, and in our own century it's going to become goulash. How does music contain this loose, runny chromaticism? By the basic principle of diatonicism, that stable relationship of tonics, and dominant, subdominants, and supertonics, and new dominants, and new tonics. We can now modulate as freely as we want, as chromatically as we want, and still have complete tonal control. This great system of tonal controls was perfected and codified by Bach, Johann Sebastian Bach, whose genius was to balance so delicately and so justly these two forces of chromaticism and diatonicism, forces that were equally powerful and presumably contradictory in nature.