\chapter{Music Structure Analysis (MSA)}

\section{The music structure analysis problem}
Audio-based Music Structural Analysis (MSA) identifies adjacent and non-overlapping musical segments in an audio signal and labels them based on similarity. Segments can occur at various time scales, from short motives to longer sections. Flat MSA focuses on non-hierarchical structures, while hierarchical MSA considers multiple time scales. However, MSA faces challenges such as boundary placement ambiguity and similarity quantification. Segments must be non-overlapping for a given hierarchical level, although some musicological approaches allow for overlap.

\subsection{Problem definition}

The musical structure analysis (MSA) challenges the definition due to subjectivity, ambiguity, and data scarcity. A flat structural analysis of an audio signal consists of a set of contiguous, non-overlapping, and labeled time intervals that span the entire duration of the audio signal. Given a musical recording of T audio samples, a flat segmentation is defined by a set of segment boundaries B, a set of k unique labels Y, and a mapping of segment starting points to labels S. This leads to the label assignment L, which assigns a label to each time point. The segment boundaries are usually sampled at a rate of 10 Hz to achieve a balance between resolution and computational efficiency.

As an art form, music is open to interpretation and can evoke different emotions and meanings in different listeners. This subjectivity makes it difficult to objectively analyze the structure of a piece of music, as different annotators may place boundaries between segments differently based on their perception and interpretation of the music. It can also have multiple structures, including hierarchical structures that make it difficult to untangle the truth.

\subsection{Segmentation principles}
The principles of musical structure analysis (MSA) were initially defined as homogeneity, novelty, and repetition, with the addition of regularity later on. To analyze the structure of a musical recording, a self-similarity matrix (SSM) is often used to visualize the degree of similarity between audio frames. The SSM results in a square matrix with the highest degree of similarity on the diagonal.

\subsubsection{Homogeneity and novelty}
Homogeneity assumes that segments are homogeneous in musical attributes, with boundaries between dissimilar segments being points of novelty. 

\subsubsection{Repetition}
Repetition assumes that segments with the same label are similar sequences, with boundaries defined by repeated sequences' start and endpoints. The structure of the music is represented using a self-similarity matrix to make its structure visually apparent. Homogeneous blocks may be challenging to subdivide, while repeated sequences appear as paths in the self-similarity matrix. MSA approaches focus on identifying these repeated sequences.

\subsubsection{Regularity}
The regularity principle in music structure analysis aims to exploit the regularities present in musical segments, such as the log-normally distributed length across the track and the tendency for the duration of two equally labeled segments to span an integer ratio of beats.

\section{Theoretical approach}
\subsection{Heinrich Schenker}

Heinrich Schenker~\cite{schenkerdocumentsonline} was a prominent music theorist who developed \textit{Schenkerian Analysis}, a method for analyzing tonal music. This method is based on the concept of "the Ursatz," a fundamental structure of three levels: foreground, middle-ground, and background. Schenker's approach emphasizes the horizontal dimension of music, specifically melody, and the relationships between individual notes and the piece's overall structure. His ideas have significantly impacted the field of music theory and analysis, and the method is still widely used today. While the approach is primarily theoretical and relies on human interpretation, it can also be approached through computer algorithms such as CNNs; the results are not the same as human interpretation.

\subsection{Schenkerian analysis}

Schenkerian analysis is a musical analysis method that examines tonal music using the theories of Heinrich Schenker. The purpose is to showcase the organic unity of a piece by highlighting the relationship between the foreground (all musical notes in the score) and the deep abstract structure known as the \textit{Ursatz}. The Ursatz, roughly the same for all tonal music, is then developed into a unique foreground in each work. The concept of tonal space, defined as the intervals between the notes of the tonic triad in the background, is crucial to the analysis. The tonal space is filled with passing and neighbor tones, creating new triads and tonal spaces for further elaboration.

Schenkerian analysis uses specialized musical notation and is either presented in a generative manner, starting from the Ursatz and ending at the score or in a reductive manner, starting from the score and reducing it to its fundamental structure. The Ursatz is presented in an arrhythmic graph, and rhythmic signs display hierarchical relationships between pitch events.

While complex and challenging, Schenkerian analysis is designed to reveal the internal coherence of a work, which ultimately lies in its tonality. The analysis's results may reflect the analyst's perceptions and intuitions.




