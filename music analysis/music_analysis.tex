\chapter{About Musical Analysis}

\section{Heinrich Schenker}

Heinrich Schenker was a prominent figure in music theory, known for developing the analytical method called "Schenkerian Analysis." This method is primarily applied to the study of tonal music. It is based on the concept of "the Ursatz," or fundamental structure, which consists of three levels: foreground, middle-ground, and background.

The foreground level is the surface structure of the musical composition, comprising the melody and harmony. The middle-ground level is the next structural level beneath the surface, while the background level represents the underlying tonal structure of the piece.

Schenker's approach to analysis postulates that all tonal music can be understood as a series of "preparations" and "resolutions" of tonal tensions. This approach emphasizes the horizontal dimension of music, specifically melody, and the relationship between individual notes and the piece's overall structure, as opposed to the vertical dimension of harmony.

Schenker's ideas have significantly impacted the field of music theory and analysis, and his method is still widely applied in the study of tonal music. Schenkerian Analysis is employed in musicology, composition, and music education and is considered a valuable tool for comprehending the structure and meaning of musical compositions.

Schenkerian Analysis can be considered a type of multi-level analysis, as it breaks down a piece of music into different levels of structure, with each level representing a different level of complexity and abstraction.

This hierarchical approach to analysis, where different levels of structure are examined, is a characteristic of multi-level analysis in general. Therefore, Schenkerian Analysis can be considered as a form of multi-level analysis, as it analyses the piece of music on multiple levels of structure and abstraction.

While Schenkerian Analysis is primarily a theoretical approach to music analysis that relies on human interpretation and notation, it is possible to use computer algorithms, such as convolutional neural networks (CNNs), to analyze music similarly.

It is important to note that even though it is possible to use CNNs for music analysis, Schenkerian Analysis is a highly theoretical and interpretive approach that relies on human expertise and interpretation. It's hard to replicate the same results using a machine learning model like CNNs.

\section{George Russell}

"The Lydian Chromatic Concept of Tonal Organization" is a theoretical treatise by George Russell, a jazz musician, and composer. First published in 1953, the book presents a system of tonal organization based on the Lydian mode. Russell asserts that the Lydian mode, characterized by its raised fourth scale degree, is the most natural and fundamental tonal system and that all other modes and scales can be derived from it. Furthermore, he claims that the Lydian mode is the only tonal system to organize the various pitch collections utilized in contemporary music appropriately.

The book provides a methodology for arranging chords and scales based on their relationship to the Lydian mode and includes a technique for improvisation and composition utilizing this system. The book's purpose is to present a new approach to music theory that would benefit jazz musicians and composers, as well as other musicians.

The book is considered a seminal work in jazz education and has significantly impacted jazz harmony and education development. Many jazz musicians and educators have studied and applied Russell's concepts in their work.

In summary, "The Lydian Chromatic Concept of Tonal Organization" is a theoretical work that posits the Lydian mode as the most fundamental and natural tonal system and organizes chords, scales, and improvisation based on this mode.