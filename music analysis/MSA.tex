\chapter{Music Structure Analysis (MSA)}

MSA encompasses various aspects of music analysis, including algorithms, current practices, and trends in technology. This field aims to comprehensively understand music structure by applying musicological theories and principles and music signal processing algorithms. MSA considers the interplay between technical and musical aspects, focusing on improving our understanding of music signals and the music itself.

In the following section, we will examine the state of MSA, including its challenges and potential for future advancements. Topics such as data representation, computational complexity, and evaluation metrics will be discussed to provide a nuanced summary of the field. Additionally, we will consider the impact of new trends and technological advancements on MSA and the directions in which the field is likely to evolve. Through this examination, we aim to provide a thorough understanding of the technical and musical aspects of MSA and its challenges.

~\cite{Nieto2020}~\cite{Chaki2021}

\section{The music structure analysis problem}

Audio-based Music Structural Analysis (MSA) identifies adjacent and non-overlapping musical segments in an audio signal and labels them based on similarity. Segments can occur at various time scales, from short motives to longer sections. Flat MSA focuses on non-hierarchical structures, while hierarchical MSA considers multiple time scales. However, MSA faces challenges such as boundary placement ambiguity and similarity quantification. Segments must be non-overlapping for a given hierarchical level, although some musicological approaches allow for overlap.

\subsection{Problem definition}

The musical structure analysis (MSA) challenges the definition due to subjectivity, ambiguity, and data scarcity. A flat structural analysis of an audio signal consists of a set of contiguous, non-overlapping, and labeled time intervals that span the entire duration of the audio signal. Given a musical recording of T audio samples, a flat segmentation is defined by a set of segment boundaries B, a set of k unique labels Y, and a mapping of segment starting points to labels S. This leads to the label assignment L, which assigns a label to each time point. The segment boundaries are usually sampled at a rate of 10 Hz to achieve a balance between resolution and computational efficiency.

As an art form, music is open to interpretation and can evoke different emotions and meanings in different listeners. This subjectivity makes it difficult to objectively analyze the structure of a piece of music, as different annotators may place boundaries between segments differently based on their perception and interpretation of the music. It can also have multiple structures, including hierarchical structures, that make it difficult to untangle the truth.

\subsection{Segmentation principles}
The principles of musical structure analysis (MSA) were initially defined as homogeneity, novelty, and repetition, with the addition of regularity later on. To analyze the structure of a musical recording, a self-similarity matrix (SSM) is often used to visualize the degree of similarity between audio frames. The SSM results in a square matrix with the highest degree of similarity on the diagonal.

\subsubsection{Homogeneity and novelty}
Homogeneity assumes that segments are homogeneous in musical attributes, with boundaries between distinct segments being points of novelty. 

\subsubsection{Repetition}
Repetition assumes that segments with the same label are similar sequences, with boundaries defined by repeated sequences' start and endpoints. The structure of the music is represented using a self-similarity matrix to make its structure visually apparent. Homogeneous blocks may be challenging to subdivide, while repeated sequences appear as paths in the self-similarity matrix. MSA approaches focus on identifying these repeated sequences.

\subsubsection{Regularity}
The regularity principle in music structure analysis aims to exploit the regularities present in musical segments, such as the log-normally distributed length across the track and the tendency for the duration of two equally labeled segments to span an integer ratio of beats.

\section{Theoretical approach}
\subsection{Music form}

Music form refers to the structural organization of a musical composition or performance. This can include elements such as the arrangement of musical units (such as rhythm, melody, and harmony) that exhibit repetition or variation, the arrangement of instruments, or the orchestration of a symphonic piece. These elements interact to shape the composition and create a meaningful musical experience for the listener. The organizational elements can be further broken down into smaller units called phrases, which express a musical idea but lack sufficient weight to stand alone. The form of a composition unfolds over time through the expansion and development of these ideas. In tonal harmony, form is primarily articulated through cadences, phrases, and periods. 

\subsection{Heinrich Schenker}

Heinrich Schenker~\cite{schenkerdocumentsonline} was a prominent music theorist who developed \textit{Schenkerian Analysis}, a method for analyzing tonal music. This method is based on the concept of "the Ursatz," a fundamental structure of three levels: foreground, middle-ground, and background. Schenker's approach emphasizes the horizontal dimension of music, specifically melody, and the relationships between individual notes and the piece's overall structure. His ideas have significantly impacted the field of music theory and analysis, and the method is still widely used today. While the approach is primarily theoretical and relies on human interpretation, it can also be approached through computer algorithms such as CNNs; the results are not the same as human interpretation.

\subsection{Schenkerian analysis}

Schenkerian analysis is a musical analysis method that examines tonal music using the theories of Heinrich Schenker. The purpose is to showcase the organic unity of a piece by highlighting the relationship between the foreground (all musical notes in the score) and the deep abstract structure known as the \textit{Ursatz}. The Ursatz, roughly the same for all tonal music, is then developed into a unique foreground in each work. The concept of tonal space, defined as the intervals between the notes of the tonic triad in the background, is crucial to the analysis. The tonal space is filled with passing and neighbor tones, creating new triads and tonal spaces for further elaboration.

Schenkerian analysis uses specialized musical notation and is either presented in a generative manner, starting from the Ursatz and ending at the score or in a reductive manner, starting from the score and reducing it to its fundamental structure. The Ursatz is presented in an arrhythmic graph, and rhythmic signs display hierarchical relationships between pitch events.

While complex and challenging, Schenkerian analysis is designed to reveal the internal coherence of a work, which ultimately lies in its tonality. The analysis's results may reflect the analyst's perceptions and intuitions.

\subsection{Aural skills}

Aural music skills are the capacity to comprehend and analyze musical compositions through listening. These skills encompass a wide range of abilities, including identifying musical elements such as rhythm, melody, harmony, timbre, and form, performing music from memory, and transcribing audio recordings into written musical scores.

In the context of music, having strong aural skills is crucial for achieving a complete understanding of the art form. It enables musicians to make knowledgeable decisions about their playing and interpretation, as well as to evaluate the playing of others critically. Additionally, aural skills are an integral part of music education, playing a role in deepening one's appreciation of music and promoting creativity and musical expression.

Aural skills can be honed through various activities, such as listening to music, singing, playing musical instruments, and analyzing music through written or visual means. Formal music education teaches aural skills through ear training, dictation, and sight-reading exercises.

Aural music skills are crucial not only for an individual's personal musical growth but also for gaining a deeper understanding of the underlying musical structure of a piece. Identifying and analyzing musical elements through listening allows one to comprehend the relationships between these elements and how they contribute to the overall musical structure. This ability to perceive and understand musical structure is fundamental to making informed decisions and interpretations.

In addition, solid aural skills enable musicians to communicate their musical ideas to others effectively, as well as to more fully comprehend the musical ideas of others. They provide a common ground for musical discussion and collaboration, allowing musicians to create and perform music at a higher level.

Moreover, aural skills also play an essential role in preserving and continuing musical traditions, as they allow musicians to accurately pass down and preserve the musical heritage of their culture. By recognizing and understanding the underlying musical structures of traditional music, musicians can maintain the authenticity of the music and ensure its continuation for future generations.

In conclusion, aural music skills are crucial for personal musical growth and for gaining a deeper understanding of the musical structure, effective musical communication and collaboration, and the preservation of musical traditions.






