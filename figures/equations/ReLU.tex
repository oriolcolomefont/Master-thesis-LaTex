The Rectified Linear Unit (ReLU) activation function is a simple yet highly effective non-linear function used in neural networks. It helps introduce non-linearity to the model and accelerates training due to its computational efficiency. Here's a brief LaTeX explanation of the ReLU function:

ReLU is a piecewise linear function that outputs the input value if it's positive and zero otherwise. Mathematically, it can be represented as:

\begin{equation}
  f(x) =
  \begin{cases}
    x & \text{if } x > 0 \\
    0 & \text{otherwise}
  \end{cases}
\end{equation}

Derivative: The derivative of the ReLU function is used during backpropagation to update the weights in the neural network. It is given by:

\begin{equation}
  f'(x) =
  \begin{cases}
    1 & \text{if } x > 0 \\
    0 & \text{otherwise}
  \end{cases}
\end{equation}

Advantages: ReLU offers several advantages over other activation functions, such as sigmoid or hyperbolic tangent (tanh). These include faster training due to reduced computational complexity and mitigating the vanishing gradient problem.

Drawbacks: Despite its benefits, ReLU has some drawbacks, such as the "dying ReLU" issue, where neurons can become inactive during training and never fire again. This problem can be addressed using variants like Leaky ReLU or Parametric ReLU.