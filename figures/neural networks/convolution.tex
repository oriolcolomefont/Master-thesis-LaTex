\begin{figure}[ht]
    \centering
    \begin{tikzpicture}[
    2d-arr/.style={matrix of nodes, row sep=-\pgflinewidth, column sep=-\pgflinewidth, nodes={draw}}
  ]
  % TikZ code here
  
  \matrix (mtr) [2d-arr] {
  0 & 1 & 1 & |[fill=orange!30]| 1 & |[fill=orange!30]| 0 & |[fill=orange!30]| 0 & 0\\
  0 & 0 & 1 & |[fill=orange!30]| 1 & |[fill=orange!30]| 1 & |[fill=orange!30]| 0 & 0\\
  0 & 0 & 0 & |[fill=orange!30]| 1 & |[fill=orange!30]| 1 & |[fill=orange!30]| 1 & 0\\
  0 & 0 & 0 & 1 & 1 & 0 & 0\\
  0 & 0 & 1 & 1 & 0 & 0 & 0\\
  0 & 1 & 1 & 0 & 0 & 0 & 0\\
  1 & 1 & 0 & 0 & 0 & 0 & 0\\
  };

  \node[below=of mtr-5-4] {$\mathbf I$};

  \node[right=0.2em of mtr] (str) {$*$};

  \matrix (K) [2d-arr, right=0.2em of str, nodes={draw, fill=teal!30}] {
    1 & 0 & 1 \\
    0 & 1 & 0 \\
    1 & 0 & 1 \\
  };
  \node[below=of K-3-2] {$\mathbf K$};

  \node[right=0.2em of K] (eq) {$=$};

  \matrix (ret) [2d-arr, right=0.2em of eq] {
  1 & 4 & 3 & |[fill=blue!80!black!30]| 4 & 1\\
  1 & 2 & 4 & 3 & 3\\
  1 & 2 & 3 & 4 & 1\\
  1 & 3 & 3 & 1 & 1\\
  3 & 3 & 1 & 1 & 0\\
  };
  \node[below=of ret-4-3] {$\mathbf{I * K}$};

  \draw[dashed, teal] (mtr-1-6.north east) -- (K-1-1.north west);
  \draw[dashed, teal] (mtr-3-6.south east) -- (K-3-1.south west);

  \draw[dashed, blue!80!black] (K-1-3.north east) -- (ret-1-4.north west);
  \draw[dashed, blue!80!black] (K-3-3.south east) -- (ret-1-4.south west);

  \foreach \i in {1,2,3} {
      \foreach \j in {4,5,6} {
          \node[font=\tiny, scale=0.6, shift={(-1.2ex,-2ex)}] at (mtr-\i-\j) {$\times \pgfmathparse{int(mod(\i+\j,2))}\pgfmathresult$};
        }
    }
\end{tikzpicture}
    
    \caption[Illustration of the 2D convolution operation.]{\small{Illustration of the 2D convolution operation on a binary input matrix $\mathbf I$ with a kernel matrix $\mathbf K$. The orange-highlighted region in $\mathbf I$ represents the portion convolved with the kernel, resulting in the blue-highlighted element in the output matrix $\mathbf{I * K}$.}}
    \label{fig: Convolution}
\end{figure}

