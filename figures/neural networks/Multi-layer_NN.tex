\begin{figure}[ht]
	\centering
	% NEURAL NETWORK with coefficients, shifted
\begin{tikzpicture}[x=2.0cm,y=1.2cm]
  \message{^^JNeural network, shifted}
  \readlist\Nnod{4,5,5,5,3} % array of number of nodes per layer
  \readlist\Nstr{n,m,m,m,k} % array of string number of nodes per layer
  \readlist\Cstr{\strut x,a^{(\prev)},a^{(\prev)},a^{(\prev)},y} % array of coefficient symbol per layer
  \def\yshift{0.5} % shift last node for dots
  
  \message{^^J  Layer}
  \foreachitem \N \in \Nnod{ % loop over layers
    \def\lay{\Ncnt} % alias of index of current layer
    \pgfmathsetmacro\prev{int(\Ncnt-1)} % number of previous layer
    \message{\lay,}
    \foreach \i [evaluate={\c=int(\i==\N); \y=\N/2-\i-\c*\yshift;
                 \index=(\i<\N?int(\i):"\Nstr[\lay]");
                 \x=\lay; \n=\nstyle;}] in {1,...,\N}{ % loop over nodes
      % NODES
      \node[node \n] (N\lay-\i) at (\x,\y) {$\Cstr[\lay]_{\index}$};
      
      % CONNECTIONS
      \ifnum\lay>1 % connect to previous layer
        \foreach \j in {1,...,\Nnod[\prev]}{ % loop over nodes in previous layer
          \draw[connect,white,line width=1.2] (N\prev-\j) -- (N\lay-\i);
          \draw[connect] (N\prev-\j) -- (N\lay-\i);
          %\draw[connect] (N\prev-\j.0) -- (N\lay-\i.180); % connect to left
        }
      \fi % else: nothing to connect first layer
      
    }
    \path (N\lay-\N) --++ (0,1+\yshift) node[midway,scale=1.5] {$\vdots$};
  }
  
  % LABELS
  \node[above=0.8,align=center,mygreen!60!black] at (N1-1.90) {input\\[-0.2em]layer};
  \node[above=0.5,align=center,myblue!60!black] at (N3-1.90) {hidden layers};
  \node[above=1.3,align=center,myred!60!black] at (N\Nnodlen-1.90) {output\\[-0.2em]layer};
  
\end{tikzpicture}
	\caption[Network graph for perceptron.]{\small{Network graph of a perceptron with $D$ input units and $C$ output units. The $l^{\text{th}}$ hidden layer contains $m^{(l)}$ hidden units. Each neuron in a layer receives input from the previous layer and computes an output value using an activation function. The output of the last layer represents the prediction or classification result.
} \cite{tikz}}
	\label{fig: multilayer color perceptron}
\end{figure}