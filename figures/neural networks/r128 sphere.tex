% A Bloch sphere of radius |a| = 1 contains all possible states of a two-state quantum system (qubit).
% Each Bloch vector fully determines a spin-1/2 density matrix.
% Used in Exercise Sheet 10 of Statistical Physics by Manfred Salmhofer (2016), available at https://janosh.dev/physics/statistical-physics.


%\usetikzlibrary{angles, quotes}

\begin{figure}
    \centering
    \scalebox{1.0}{
    \begin{tikzpicture}

  % Define radius
  \def\r{3}

  % Bloch vector
  \draw (0, 0) node[circle, fill, inner sep=1] (orig) {} -- (\r/3, \r/2) node[circle, fill, inner sep=0.7, label=above:$\vec{a}$] (a) {};
  \draw[dashed] (orig) -- (\r/3, -\r/5) node (phi) {} -- (a);

  % Sphere
  \draw (orig) circle (\r);
  \draw[dashed] (orig) ellipse (\r{} and \r/3);

  % Axes
  \draw[->] (orig) -- ++(-\r/5, -\r/3) node[below] (x1) {$x_1$};
  \draw[->] (orig) -- ++(\r, 0) node[right] (x2) {$x_n \in \mathbb{R}$ $2 \leq n \leq 127$};
  \draw[->] (orig) -- ++(0, \r) node[above] (x3) {$x_128$};

\end{tikzpicture}}
    \caption[128-dimensional hypersphere]{\small{A representation of a 128-dimensional hypersphere in the Euclidean space $\mathbb{R}^{128}$. Due to the limitations of visualizing high-dimensional objects in two or three dimensions, this figure offers a simplified and abstract depiction of the hypersphere.}}
    \label{fig:my_label}
\end{figure}