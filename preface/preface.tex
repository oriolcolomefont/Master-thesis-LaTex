%\addcontentsline{toc}{chapter}{Acknowledgement}
\begin{preface}
\pagenumbering{gobble}% Remove page numbers (and reset to 1)

Music and science have always been my two greatest passions. When the opportunity to work at Epidemic Sound presented itself, it was as if the universe had conspired to bring my interests together, offering me the chance to explore the fascinating Music Information Retrieval (MIR) field. With the support and encouragement of my loved ones, I took a leap of faith and moved to Stockholm to embark on this exciting journey.

This work aims to help untangle the complexities of music and offer a more musically-driven perspective to MIR. I aim to contribute to a deeper understanding of the intricate relationships within musical data by bridging the gap between music theory and computational analysis.

I am deeply grateful to the Epidemic Sound team for their guidance, expertise, and camaraderie throughout this project. I would also like to extend my heartfelt appreciation to my family for their unwavering support and belief in my abilities.

My motivation for writing this piece stems from a desire to grow as a musician, programmer, professional, and individual. I believe that by delving into the world of MIR, I can expand my horizons while making a meaningful contribution to the field.

This work is intended for any curious human being, regardless of their background in music or computer science. I hope it will inspire others to explore the captivating intersection of these two disciplines.

The scope of this work encompasses a range of MIR tasks and methodologies, as well as discussions on the challenges and opportunities that arise within the field. Despite the ambitious nature of this project, I acknowledge that time and my academic background limitations may have constrained the depth of my analysis. Nevertheless, I hope this work will serve as a starting point for further exploration and inspire new ideas in the realm of MIR.

\newpage
\end{preface}