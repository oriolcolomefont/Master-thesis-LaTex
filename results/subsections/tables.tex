
As mentioned in Section \ref{section:Datasets and metrics}, the evaluation results were obtained using standard and commonly used MIR tools, frameworks, and metrics. These include the MSAF \cite{MSAF} implementation for segmentation algorithms, the SALAMI dataset \cite{Smith2011DESIGNANNOTATIONS} for evaluation ground truth, and $mireval$ Python package \cite{RaffelMir_eval:METRICS} for metric computation.

Table \ref{ta:results} provide a comparative analysis across four features using three segmentation algorithms. The results reveal that the performance of the algorithms varies depending on the feature type. For instance, the CQT feature exhibits the highest precision (0.570), recall (0.339), and f-measure (0.353) when processed with the Foote algorithm. Nevertheless, this table proves that our embeddiograms can achieve competitive performance compared to traditional handcrafted signal processing methods by learning only from unlabeled audio files.

These results are also represented in Figure \ref{fig:boxplotmetrics}, to understand the metrics' distribution for each feature visually. The high outliers suggest that there is significant variability in the performance of the model. While the model performs exceptionally well in some instances, it performs average in most cases.

Table \ref{tab:comparison_table} and Figure \ref{fig:enter-label} compare the current study's boundary detection F-measure results with those of previous studies using unsupervised methods. The most accurate result was reported in 2019 by McCallum \cite{deepfeaturesegment}, who used a CNN on CQT and achieved an F-measure of 0.535. On the other hand, our study's approach yielded an F-measure of 0.288. Numbers show that our unsupervised method is competitive with research conducted a decade ago, trailing behind the current state of the art. However, this is a promising starting point, as the unsupervised nature offers ample and almost effortless opportunities for enhancement.

%%%%%%%%%%%%%%%%%%%%%%%%%%%%

\newsavebox\embeddiobSF
\begin{lrbox}{\embeddiobSF}
   $\begin{aligned}
     \textbf{0.333} & \quad 0.280 & \quad 0.288
    \end{aligned} $
\end{lrbox}

\newsavebox\embeddiobFoote
\begin{lrbox}{\embeddiobFoote}
   $\begin{aligned}
     0.275 & \quad 0.318 & \quad 0.280
    \end{aligned} $
\end{lrbox}

\newsavebox\embeddiobCNMF
\begin{lrbox}{\embeddiobCNMF}
   $\begin{aligned}
     0.248 & \quad 0.296 & \quad 0.254
    \end{aligned} $
\end{lrbox}

\newsavebox\pcpSF
\begin{lrbox}{\pcpSF}
   $\begin{aligned}
     0.311 & \quad 0.324 & \quad 0.305
    \end{aligned} $
\end{lrbox}

\newsavebox\pcpFoote
\begin{lrbox}{\pcpFoote}
   $\begin{aligned}
     0.288 & \quad \textbf{0.331} & \quad 0.295
    \end{aligned} $
\end{lrbox}

\newsavebox\pcpCNMF
\begin{lrbox}{\pcpCNMF}
   $\begin{aligned}
     0.228 & \quad 0.310 & \quad 0.250
    \end{aligned} $
\end{lrbox}

\newsavebox\tonnetzSF
\begin{lrbox}{\tonnetzSF}
   $\begin{aligned}
     0.312 & \quad 0.312 & \quad 0.300
    \end{aligned} $
\end{lrbox}

\newsavebox\tonnetzFoote
\begin{lrbox}{\tonnetzFoote}
   $\begin{aligned}
     0.272 & \quad 0.317 & \quad 0.280
    \end{aligned} $
\end{lrbox}

\newsavebox\tonnetzCNMF
\begin{lrbox}{\tonnetzCNMF}
   $\begin{aligned}
     0.212 & \quad 0.306 & \quad 0.237
    \end{aligned} $
\end{lrbox}

\newsavebox\cqtSF
\begin{lrbox}{\cqtSF}
   $\begin{aligned}
     0.311 & \quad \textbf{0.339} & \quad \textbf{0.312}
    \end{aligned} $
\end{lrbox}

\newsavebox\cqtFoote
\begin{lrbox}{\cqtFoote}
   $\begin{aligned}
     \textbf{0.570} & \quad 0.311 & \quad \textbf{0.353}
    \end{aligned} $
\end{lrbox}

\newsavebox\cqtCNMF
\begin{lrbox}{\cqtCNMF}
   $\begin{aligned}
     \textbf{0.296} & \quad \textbf{0.311} & \quad \textbf{0.287}
    \end{aligned} $
\end{lrbox}

%%%%%

\begin{table}[ht]
  \centering
  \begin{adjustbox}{width=\textwidth}
  \begin{threeparttable}
    \begin{tabular}{c|c|c|c|c} 
\toprule
     \textbf{Feature} & \textbf{SF} & \textbf{Foote} & \textbf{CNMF} 
     \\ \midrule 
     PCP & \usebox{\pcpSF} & \usebox{\pcpFoote} & \usebox{\pcpCNMF} \\\hline
     Tonnetz & \usebox{\tonnetzSF} & \usebox{\tonnetzFoote} & \usebox{\tonnetzCNMF} \\\hline
     CQT & \usebox{\cqtSF} & \usebox{\cqtFoote} & \usebox{\cqtCNMF} \\\hline
     Embeddiogram & \usebox{\embeddiobSF} & \usebox{\embeddiobFoote} & \usebox{\embeddiobCNMF} \\
     \bottomrule
    \end{tabular}
    \caption[Metric comparison: audio features and segmentation algorithms]{Comparison of precision (left column), recall (middle column), and f-measure (right column) metrics for different features using the Structural Feature (SF)\cite{sf}, Checkerboard-like Kernel (Foote) \cite{Foote2000AutomaticNovelty}, and Convex Non-negative Matrix Factorization (CNMF) \cite{NietoCONVEXIDENTIFICATION} algorithms on the SALAMI dataset. The sliding windowed segments across the input signal is 4 seconds long.}\label{ta:results}
  \end{threeparttable}
  \end{adjustbox}
\end{table}

%%%%%%%%%%%%%%%%%%%%%%%%%%%

\begin{figure}
    \centering
    \includegraphics[width=\textwidth]{figures/images/boxplot.png}
    \caption[Metric comparison for different audio features.]{\small{Boxplot visual comparison of different features' average precision, recall, and f-measure. Sliding windowed segments across the audio signal input signal is 4 seconds.}}
    \label{fig:boxplotmetrics}
\end{figure}

%%%%%%%%%%%%%%%%%%%%%%%%%%%%

%%%%%%%%%%%%%%%%%%%%%%%%%%%%

\begin{table}[ht]
\centering
\small
\begin{tabularx}{\textwidth}{>{\centering\arraybackslash}p{4.5cm}>{\centering\arraybackslash}X>{\centering\arraybackslash}X>{\centering\arraybackslash}X}
\toprule
\textbf{Training dataset [Ref]}  & \textbf{Precision} & \textbf{Recall} & \textbf{F-measure} \\
\midrule
\addlinespace
GTZAN \cite{GTZAN} & 0.228 & 0.171 & 0.185 \\
\addlinespace
MSD \cite{MSD} &  \textbf{0.333} &  \textbf{0.280} & \textbf{0.288} \\
\addlinespace
\bottomrule
\end{tabularx}
\caption[GTZAN-trained versus MSD-trained embeddiograms]{\small{Comparison of precision, recall, and F-measure for GTZAN-trained versus MSD-trained embeddiograms on SALAMI dataset computed using the Structural Feature (SF)\cite{sf} algorithm.}}
\label{tab:GTZAN-MSD-embed}
\end{table}

%%%%%%%%%%%%%%%%%%%%%%%%%%%%

Table \ref{tab:GTZAN-MSD-embed} compares the performance of the Structural Feature (SF) algorithm on the SALAMI dataset, utilizing two distinct sets of embeddiograms as input features. These embeddiograms were generated with the neural network trained using the GTZAN and MSD datasets, respectively, and the results show a clear advantage when trained on the MSD dataset compared to the GTZAN dataset. Specifically, the precision, recall, and f-measure are all higher for the MSD-trained algorithm.

%%%%%%%%%%%%%%%%%%%%%%%%%%%%

\begin{table}[ht]
\begin{threeparttable}
\centering
\small
\begin{tabularx}{\textwidth}{
  >{\centering\arraybackslash}p{4.5cm}
  >{\centering\arraybackslash}X
  >{\centering\arraybackslash}X
  >{\centering\arraybackslash}X}
\toprule
\thead{\textbf{Authors [Ref], Year}} & \thead{\textbf{Input}\tnote{1}} & \thead{\textbf{Method}} & \thead{\textbf{F-measure}} \\
\midrule
Turnbull et al. \cite{Turnbull2007ABOOSTING}, 2007 & MFCCs, chromas, spectrogram & Boosted Decision Stump  & 0.378 \\
\addlinespace
Kaiser et al. \cite{27}, 2012 & SSM & Novelty measure  & 0.286 \\
\addlinespace
Sargent et al. \cite{34}, 2011 & MFCCs, chromas & Viterbi  & 0.356 \\
\addlinespace
McFee \& Ellis \cite{20}, 2013 & MLS & Fisher’s Linear Discriminant  & 0.317 \\
\addlinespace
Nieto \& Bello \cite{28}, 2014 & MFCCs, chromas & Checkerboard-like kernel  & 0.299 \\
\addlinespace
Cannam et al. \cite{29}, 2015 & Timbre-type histograms & HMM  & 0.213 \\
\addlinespace
Nieto \cite{30}, 2016 & CQT Spectrogram & Linear Discriminant Analysis  & 0.299 \\
\addlinespace
Cannam et al. \cite{29}, 2017 & Timbre-type histograms & HMM  & 0.212 \\
\addlinespace
Ullrich et. al \cite{22}, 2014 & MLS & CNN  & 0.465 \\
\addlinespace
Grill \& Schlüter \cite{4}, 2015 & MLS, SSLMs & CNN  & 0.523 \\
\addlinespace
Grill \& Schlüter \cite{Grill2015MusicAnnotations}, 2015 & MLS, PCPs, SSLMs & CNN  & 0.508 \\
\addlinespace
Hadria \& Peeterss \cite{35}, 2017 & MLS, SSLMs & CNN  & 0.291 \\
\addlinespace
McCallum \cite{deepfeaturesegment}, 2019 & CQT & CNN  & \textbf{0.535} \\
\addlinespace
Ours, 2023 & Raw waveforms & CNN  & 0.288 \\
\bottomrule
\end{tabularx}
\caption[Baseline. State-of-the-art comparison table.]{Previous studies' boundary detection f-measure results using unsupervised methods for a 0.5s time-window tolerance. Only the top-performing algorithm for each year on the SALAMI dataset is displayed. This table has been extended from \cite{Hernandez-Olivan2021MusicFeatures}.}
\label{tab:comparison_table}
\begin{tablenotes}\footnotesize
\item[1] \textbf{Legend:} SSM: Self-Similarity Matrix, MLS: Mel Spectrogram, MFCC: Mel-Frequency Cepstral Coefficient, CQT: Constant Q-Transform, PCP: Pitch Class Profile, SSLM: Self-Similarity Lag Matrix.
\end{tablenotes}
\end{threeparttable}
\end{table}

%%%%%%%%%%%%%%%%%%%%

\begin{figure}
    \centering
    \includegraphics[width=\textwidth]{figures/images/fmeasuregraphs.png}
    \caption[Baseline. State-of-the-art graph.]{Previous studies' boundary detection f-measure results using unsupervised methods for a 0.5s time-window tolerance. Only the top-performing algorithm for each year on the SALAMI dataset is displayed. This figure has been extended from \cite{Hernandez-Olivan2021MusicFeatures}.}
    \label{fig:enter-label}
\end{figure}

%%%%%%%%%%%%%%%%%%%%%%%


The GitHub repository containing all the code needed to run the experiments can be found \href{https://github.com/oriolcolomefont/Master-Thesis.git}{HERE}.


