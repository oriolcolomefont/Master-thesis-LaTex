\chapter{Results}

XXXXXXXXXXXXXXXXX

\section{Tables and graphics}

XXXXXXXXXXXXXXXXX

Extend table by Hernandez-Olivan, C., Beltran, J. R. \& Diaz-Guerra, D. 
\cite{Hernandez-Olivan2021MusicFeatures}

\begin{table}[h]
\centering
\small
\begin{tabularx}{\textwidth}{>{\raggedright\arraybackslash}p{5cm}XXXXX}
\toprule
\thead{\textbf{Year, Authors, Ref}} & \thead{\textbf{Algorithm}} & \thead{\textbf{Input}} & \thead{\textbf{Method}} & \thead{\textbf{F-measure SALAMI}} \\
\midrule
\addlinespace
2009, Paulus \& Klapuri, [24] & PK & MFCCs, chromas & Fitness function  & - \\
\addlinespace
2010, Mauch et al., [25] & MND1 & MFCCs, Discrete Cepstrum & HMM  & - \\
\addlinespace
2011, Sargent et al., [26]& SB-VRS1 & Chords estimation & Viterbi  & - \\
\addlinespace
2012, Kaiser et al., [27]& KSP2 & SSM & Novelty measure  & 0.286 \\
\addlinespace
2013, McFee \& Ellis, [20] & MP2 & MLS & Fisher’s Linear Discriminant  & 0.317 \\
\addlinespace
2014, Nieto \& Bello, [28] & NB1 & MFCCs + chromas & Checkerboard-like kernel  & 0.299 \\
\addlinespace
2015, Cannam et al., [29] & CC1 & Timbre-type histograms & HMM  & 0.213 \\
\addlinespace
2016, Nieto, [30] & ON2 & Constant-Q Transform Spectrogram & Linear Discriminant Analysis  & 0.299 \\
\addlinespace
2017, Cannam et al., [29] & CC1 & Timbre-type histograms & HMM  & 0.212 \\
\addlinespace
2007, Turnbull et al., [19] & - & MFCCs, chromas, spectrogram & Boosted Decision Stump  & 0.378 \\
\addlinespace
2011, Sargent et al., [34] & - & MFCCs, chromas & Viterbi  & 0.356 \\
\addlinespace
2014, Ullrich et. al, [22] & - & MLS & CNN  & 0.465 \\
\addlinespace
2015, Grill \& Schlüter, [4] & - & MLS + SSLMs & CNN  & 0.523 \\
\addlinespace
2015, Grill \& Schlüter, [5] & - & MLS + PCPs + SSLMs & CNN  & 0.508 \\
\addlinespace
2017, Hadria \& Peeters, [35] & - & MLS + SSLMs & CNN  & 0.291 \\
\bottomrule
\end{tabularx}
\caption{Your Caption}
\label{tab:my_label}
\end{table}


\begin{table}[!ht]
\renewcommand{\arraystretch}{1.50}
\caption[Table]{This is an example of a table and its caption.}
\label{tablePCA}
\centering
\begin{tabular}{| c | c | c | c | c |}
\hline
\bfseries PCA & \bfseries Residual mean & \bfseries XXX & \bfseries XXX & \bfseries XXX \\
\hline\hline
Original PCA & 0.1267 & XXX & XXX & XXX  \\
\hline
PCA on Centroid 1 & 0.1249 & XXX & XXX  & XXX\\
\hline
PCA on Centroid 2 & 0.1214  & XXX & XXX  & XXX\\
\hline
\end{tabular}
\end{table}

\newpage


