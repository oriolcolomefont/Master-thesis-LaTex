%\addcontentsline{toc}{chapter}{Abstract}

\begin{abstract}
\pagenumbering{gobble}

This thesis posits the existence of invariant high-level musical concepts that persist regardless of changes in sonic qualities, akin to the symbolic domain where essence endures despite varying interpretations through different performances, instruments, and styles, among many other, almost countless variables.

In collaboration with Epidemic Sound AB and the Music Technology Group (MTG) at Universitat Pompeu Fabra (UPF), we used self-supervised contrastive learning to uncover the underlying structure of Western tonal music by learning deep audio features to improve unsupervised music boundary detection. We applied deep convolutional neural networks and a triplet loss function to identify abstract and semantic high-level musical elements without relying on their sonic qualities. This way, we replaced traditional acoustic features with deep audio embeddings, paving the way for sound-agnostic and content-sensitive music representation for boundary detection.

Our cognitively-based approach for learning embeddings focuses on using full-resolution data and preserving high-level musical information which unfolds in the time domain. A key component in our methodology is the use of triplet networks, which provide an effective way of understanding and preserving the relative distances between different pieces of music. This model structure is integral to our work, allowing us to identify and maintain the nuanced relationships within our musical data. Drawing upon our domain expertise, we developed robust transformations to encode heuristic musical concepts that should remain constant. This novel approach aims to reconcile music and machine learning, enhancing machine listening models' efficacy through deep learning and triplet networks. 

Preliminary results suggest that, while not outperforming state-of-the-art results, our musically-informed technique has significant potential for boundary detection tasks and, most likely, nearly all MIR downstream tasks that are not purely sonic-based.

While music-motivated audio embeddings appear promising, delivering competitive results with room for improvement and potentially adaptable to other tasks constrained by data scarcity, the question remains if such general-purpose audio representation can mimic human hearing.

\bigskip
Keywords: MIR; Music Structure Analysis; deep audio embeddings; audio representations; representation learning; embeddings; transfer learning; multi-task learning; multi-modal learning; aural Skills

\newpage
\end{abstract}