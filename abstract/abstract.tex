%\addcontentsline{toc}{chapter}{Abstract}

\begin{abstract}
\pagenumbering{gobble}

Music, an integral part of human culture, offers a diverse yet complex field of study due to its intricate styles, mutable ground truth, and subjective nature. This research, conducted in collaboration with Epidemic Sound AB and the Music Technology Group at Universitat Pompeu Fabra, employs self-supervised learning and contrastive learning of musical representation to uncover the fundamental tonal structure of Western music. 

Embracing a subtle novel approach, this process is facilitated through the application of Siamese networks with a triplet loss, mimicking human aural skills in interpreting music to discern musical elements, irrespective of the sonic qualities. The study replaces traditional acoustical features with deep audio embeddings, to compute high-level, sound-agnostic, and content-sensitive music identification.

Results XXXXXX

We strongly believe that this approach to learning music-informed embeddings can benefit almost every MIR downstream task. The potential implications of this research include advancements in intelligent music recommendation systems and efficient intellectual property rights enforcement, ultimately promoting innovation across the music industry.

\bigskip
Keywords: Music; Music Theory; Music Information Retrieval; Music Structure Analysis; Machine Learning; Deep Learning; Self-Supervised Learning;

\newpage
\end{abstract}