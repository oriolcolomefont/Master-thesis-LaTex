%\addcontentsline{toc}{chapter}{Abstract}

\begin{abstract}
\pagenumbering{gobble}

This thesis posits the existence of high-level musical concepts invariant to sonic qualities that evolve and unfolds through time. This idea parallels the nature of the symbolic domain, which maintains its essence despite being interpreted through diverse performances using various instruments and styles.

This research, conducted in collaboration with Epidemic Sound AB and the Music Technology Group (MTG) at Universitat Pompeu Fabra (UPF), employs self-supervised contrastive learning of musical representation to uncover the fundamental structure of Western tonal music. 

Embracing a subtle novel approach, this process is facilitated by using deep neural networks with a triplet loss, somewhat mimicking human aural skills in interpreting music to discern abstract and semantic musical elements, irrespective of the sonic qualities. The study replaces traditional acoustical features with deep audio embeddings to compute high-level, sound-agnostic, and content-sensitive music identification.

Results XXXXXX TBC

The preliminary results suggest that our approach to learning musically-informed embeddings holds significant potential for nearly all MIR downstream tasks. Music-motivated embeddings represent a promising technique, adaptable potentially to other tasks hindered by data scarcity. This method could significantly influence advancements in intelligent music recommendation systems and the efficient enforcement of intellectual property rights. 

\bigskip
Keywords: Music Information Retrieval; Music Structure Analysis; Deep Audio Embeddings, Aural Skills

\newpage
\end{abstract}