%\addcontentsline{toc}{chapter}{Abstract}

\begin{abstract}
\pagenumbering{gobble}

This thesis asserts the existence of high-level musical concepts that, while evolving and unfolding over time, remain invariant to changes in sonic qualities. This notion echoes the symbolic domain's characteristics, wherein the essence persists regardless of varied interpretations through diverse performances, instruments, and styles.

In collaboration with Epidemic Sound AB and the Music Technology Group (MTG) at Universitat Pompeu Fabra (UPF), we employed self-supervised contrastive learning of musical representation to decipher the underlying structure of Western tonal music.

Taking a subtle yet novel approach, we utilized deep neural networks with a triplet loss function to facilitate this process. This strategy mirrors human aural skills in music interpretation, discerning abstract and semantic musical elements irrespective of their sonic qualities. This research supplants traditional acoustical features with deep audio embeddings, enabling high-level, sound-agnostic, and content-sensitive music identification.

Our method of learning embeddings is significant, emphasizing the use of full-resolution data to maintain the integrity of high-level musical information. Leveraging our expertise in the music domain, we construct aggressive transformations to embed heuristic musical concepts. With these innovative strides, we strive to bridge the gap between music and machine learning, improving the efficacy of machine listening models.

Preliminary results indicate that our musically-informed approach to learning embeddings carries the substantial potential for boundary detection tasks and, as our intuition suggests, possibly for nearly all Music Information Retrieval (MIR) downstream tasks. Music-motivated embeddings emerge as a promising technique, potentially adaptable to other tasks challenged by data scarcity.

\bigskip
Keywords: Music Information Retrieval; Music Structure Analysis; Deep Audio Embeddings, Aural Skills

\newpage
\end{abstract}