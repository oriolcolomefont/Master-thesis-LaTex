\section{Model architecture and training strategy}
Piaget's theory of cognitive development \cite{Huitt2003PiagetsDevelopment} posits that children acquire knowledge from sensory and motor experiences in their early stages, from birth to around 18 months, during the \textit{sensorimotor stage}. This stage sees the emergence of early representational thought through basic actions such as sucking, grasping, looking, and listening. As children progress through different developmental stages until adolescence and adulthood, their reasoning progressively moves toward abstract ideas and deductive logic. Schemas, higher-order cognitive structures hypothesized to underlie many aspects of human knowledge and skill, emerge during this process, which Piaget interprets through an equilibration mechanism that balances new information with old knowledge. This mechanism involves \textit{assimilation} and \textit{accommodation}, which entail taking in further information to fit pre-existing schemas and modifying existing schemas resulting from the latest information. In the context of learning, complex structures are based on simpler ones, and knowledge transfer occurs when dynamic systems capable of generalization are created. \cite{audioselfsupsurvey}

There are similarities between assimilation and accommodation in Piaget's theory of cognitive development and backpropagation in DL, both of which involve modifying existing structures based on new information. However, it should be noted that these processes' underlying mechanisms and contexts are distinct. While assimilation and accommodation involve modifying mental structures to incorporate new information, backpropagation entails adjusting neural network weights to improve output predictions. Assimilation and accommodation occur in cognitive development, while backpropagation is specific to machine learning. Thus, while these processes have broad similarities, they are fundamentally different and cannot be directly compared in detail.

\subsection{Contrastive Self-Supervised Learning of Musical Representations (CSSLMR)}

Contrastive Self-Supervised Learning of Musical Representations (CSSLMR) is an approach that combines the principles of Self-Supervised Learning (SSL) \cite{audioselfsupsurvey} and Contrastive Learning of Musical Representations (CLMR) \cite{CLMR2021} to learn discriminative and valuable representations of music without relying on explicit labels. By leveraging the structure and content of music and contrasting different augmentations or transformations of the same musical piece against randomly (or not) selected pieces, CSSLMR models can effectively capture the underlying structure and known invariance in the data.

The CSSLMR methodology involves the following steps:

\begin{enumerate}
\item \textbf{Data Augmentation:} Create different augmentations or transformations of the same musical piece as positive pairs and use other randomly selected pieces as antagonistic pairs.
\item \textbf{Contrastive Learning:} Train a neural network to minimize a contrastive loss function, encouraging the model to produce similar positive and dissimilar representations for antagonistic pairs.
\item\textbf{Latent Representation:} Utilize the learned representations for various MIR downstream tasks, such as music genre classification or similarity, both in the symbolic and time domains.
\end{enumerate} 

\subsubsection{Strengths}

CSSLMR inherits the strengths of both SSL and CLMR:

\begin{enumerate}
\item Reduces dependency on labeled data: by leveraging inductive bias (assuming a general invariance across the data samples), CSSL can learn from large amounts of unlabeled data.
\item Encourages feature extraction: CSSLMR promotes learning practical and discriminative representations.
\end{enumerate} 

\section{Model architecture}

We propose using a Triple Siamese Network to train a model for music similarity retrieval \cite{contentmusicsimtriplet2020}. Specifically, we aim to minimize the loss function between an anchor, a positive, and a negative sample by using online triplet mining \cite{Sikaroudi2020OfflinePatches}.

Using the proposed triple siamese network and online triplet mining, we aim to train a model that can effectively distinguish between similar and dissimilar samples while considering their underlying composition and production texture.


\subsection{Encoder}
\subsubsection{SampleCNN}

The SampleCNN model \cite{CLMR2021} takes in 1D input data with a single channel. The first layer is a 1D convolutional layer with a kernel size of 3, a stride of 3, and 128 output channels. It is followed by batch normalization and ReLU activation.

After the initial layer, there are nine hidden layers with varying kernel sizes, strides, and channels. Each hidden layer consists of the following:

\begin{itemize}
    \item A 1D convolutional layer with the specified number of input and output channels, kernel size, a stride of 1, and padding of 1.

    \begin{equation}
y_{c_o}(n) = \sum_{c_i=1}^{C_{\text{in}}} \sum_{k=-\lfloor K/2 \rfloor}^{\lfloor K/2 \rfloor} x_{c_i}(n - k) \cdot w_{c_o, c_i}(k)
\end{equation}

Where:

$y_{c_o}(n)$ is the output for the $c_o$-th output channel at position $n$
$C_{\text{in}}$ is the number of input channels
$x_{c_i}(n)$ is the input for the $c_i$-th input channel at position $n$
$w_{c_o, c_i}(k)$ is the kernel weight for the $c_o$-th output channel and $c_i$-th input channel at position $k$
$K$ is the kernel size
The stride is set to 1 and padding is set to 1 (equal padding $\lfloor K/2 \rfloor$ on both sides)
s
    \item Batch normalization. We process a batch of data containing anchor, positive, and negative samples. It pads the waveforms to the same length, performs online triplet mining to find the hardest negatives, and returns the padded anchors, positives, and hardest negatives.

    Let $A$, $P$, and $N$ represent the anchor, positive, and negative samples. The maximum length in the batch is given by:

    \begin{equation}
L_{\text{max}} = \max_{\text{item} \in \text{batch}}(\max(\text{length}(A_{\text{item}}), \text{length}(P_{\text{item}}), \text{length}(N_{\text{item}})))
\end{equation}

    The anchor-positive distance and the anchor-negative distance are computed as the Euclidean distance between the corresponding samples:

    \begin{equation}
D_{\text{AP}} = \sqrt{\sum_{i} (A_i - P_i)^2}
\end{equation}

    The "hardest" negative index for each anchor-positive pair is determined by maximizing the difference between the anchor-negative and anchor-positive distances:

    \begin{equation}
D_{\text{AN}} = \sqrt{\sum_{i} (A_i - N_i)^2}
\end{equation}

    
    \item ReLU activation. The Rectified Linear Unit (ReLU) activation function is a simple yet highly effective non-linear function used in neural networks. It helps introduce non-linearity to the model and accelerates training due to its computational efficiency.

    
    \item Max pooling with kernel size and stride equal to the stride value of the current layer.
\end{itemize}

\begin{tabular}{|c|c|c|}
\hline
\textbf{Layer} & \textbf{Input Channels} & \textbf{Output Channels} \\
\hline
1 & 128 & 128 \\
\hline
2 & 128 & 128 \\
\hline
3 & 128 & 256 \\
\hline
4 & 256 & 256 \\
\hline
5 & 256 & 256 \\
\hline
6 & 256 & 256 \\
\hline
7 & 256 & 256 \\
\hline
8 & 256 & 256 \\
\hline
9 & 256 & 512 \\
\hline
\end{tabular}

After the hidden layers, there's another 1D convolutional layer with 512 input channels, 512 output channels, a kernel size of 3, a stride of 1, and padding of 1. This is followed by batch normalization and ReLU activation.

The output of the final convolutional layer is passed through an average pooling operation across the temporal dimension (dim=2).

If the model is "supervised", dropout with a rate of 0.5 is applied; a regularization technique randomly drops out neurons during training to prevent overfitting, with a 0.5 rate meaning half of the neurons are dropped out.

\def\layersep{2}
\def\nodesep{1.5}

\begin{figure}[h]
    \centering
    \scalebox{0.7}{
    \begin{tikzpicture}[node/.style={circle, draw, thick}]
  % TikZ code here
  \foreach \y in {1,...,5}{
      \node[node] (i\y) at (0,\nodesep*\y) {};
      \node[node, right=\layersep of i\y] (h1\y) {};
      \node[node, right=\layersep of h1\y] (h2\y) {};
    }

  \node[node, right=\layersep of h22] (o1) {};
  \node[node, right=\layersep of h24] (o2) {};

  \foreach \source in {1,...,5}
  \foreach \dest in {1,...,5}{
      \path[-stealth, thick] (i\source) edge (h1\dest);
      \path[-stealth, thick] (h1\source) edge (h2\dest);
    }
  \foreach \source in {1,...,5}
  \foreach \dest in {1,2}
  \draw[-stealth, thick] (h2\source) -- (o\dest);

  \draw[-stealth, thick] (8.3,3*\nodesep) -- node[above,font=\Large\bfseries] {dropout} (10.3, 3*\nodesep);

  % Boundary

  \foreach \y in {1,...,5}
  \node[node, right=15em of h2\y] (di\y) {};

  \node[red,font=\huge] at (di1) {$\times$};
  \node[red,font=\huge] at (di3) {$\times$};

  \foreach \y in {1,...,5}
  \node[node, right=\layersep of di\y] (dh1\y) {};

  \node[red,font=\huge] at (dh11) {$\times$};
  \node[red,font=\huge] at (dh13) {$\times$};
  \node[red,font=\huge] at (dh14) {$\times$};

  \foreach \y in {1,...,5}
  \node[node, right=\layersep of dh1\y] (dh2\y) {};

  \node[red,font=\huge] at (dh22) {$\times$};
  \node[red,font=\huge] at (dh24) {$\times$};

  \node[node, right=\layersep of dh22] (do1) {};
  \node[node, right=\layersep of dh24] (do2) {};

  \foreach \source in {2,4,5}
  \foreach \dest in {2,5}
  \draw[-stealth, thick] (di\source) -- (dh1\dest);

  \foreach \source in {2,5}
  \foreach \dest in {1,3,5}
  \draw[-stealth, thick] (dh1\source) -- (dh2\dest);

  \foreach \source in {1,3,5}
  \foreach \dest in {1,2}
  \draw[-stealth, thick] (dh2\source) -- (do\dest);

\end{tikzpicture}
    }
    \caption{\small{High-level schematic dropout representation}}
    \label{figure: dropout}
\end{figure}

Finally, a linear layer with 512 input features and 128 output dimension features is applied to generate the embeddings.

\begin{tikzpicture}[
  node distance=0.5cm,
  block/.style={draw, rectangle, fill=blue!20, text width=1.5cm, align=center, minimum height=1cm},
  arrow/.style={->, >=stealth, line width=0.5mm}
]

% Input
\node (input) [block] {Input};

% Conv 1
\node (conv1) [block, right=of input] {Conv1D\\128};
\draw[arrow] (input) -- (conv1);

% BatchNorm 1 and ReLU 1
\node (bn1) [block, right=of conv1] {BatchNorm\\ReLU};
\draw[arrow] (conv1) -- (bn1);

% Hidden Layers
\node (hidden1) [block, right=of bn1] {Hidden Layer 1};
\draw[arrow] (bn1) -- (hidden1);

\node (hidden2) [block, right=of hidden1] {Hidden Layer 2};
\draw[arrow] (hidden1) -- (hidden2);

\node (hidden3) [block, right=of hidden2] {Hidden Layer 3};
\draw[arrow] (hidden2) -- (hidden3);

\node (dots) [right=of hidden3] {$\cdots$};
\draw[arrow] (hidden3) -- (dots);

% Conv N
\node (convN) [block, right=of dots] {Conv1D\\512};
\draw[arrow] (dots) -- (convN);

% BatchNorm N and ReLU N
\node (bnN) [block, right=of convN] {BatchNorm\\ReLU};
\draw[arrow] (convN) -- (bnN);

% Average Pooling
\node (avgpool) [block, right=of bnN] {Avg Pool};
\draw[arrow] (bnN) -- (avgpool);

% Dropout (optional)
\node (dropout) [block, right=of avgpool, dashed] {Dropout (optional)};
\draw[arrow, dashed] (avgpool) -- (dropout);

% Fully Connected
\node (fc) [block, right=of dropout] {Linear\\Out\_dim};
\draw[arrow] (dropout) -- (fc);

\end{tikzpicture}


We have adapted the original architecture for our experiments to our purpose and needs—the weights initialization and the final average pooling step.

\subsection{Audio augmentation and transformation pipeline}
\subsubsection{Positive sample generation}
Let $W$ be the input waveform and $W_p$ be the generated positive waveform. The pipeline applies audio effects and adds noise with a random SNR. Define the following random variables for the effects:

\begin{equation}
gain \sim \text{Uniform}(-12, 0)
\end{equation}
\begin{equation}
pitch \sim \text{Uniform}(-1200, 1200)
\end{equation}
\begin{equation}
reverb\_params \sim \text{Uniform}(0, 100)^3
\end{equation}
\begin{equation}
chorus\_params = \begin{cases}
\text{Uniform}(0.1, 1.0), & i=1,2, \\
\text{Uniform}(20, 55), & i=3, \\
\text{Uniform}(0.1, 0.9), & i=4, \\
\text{Uniform}(0.1, 2.0), & i=5, \\
\text{Uniform}(2, 5), & i=6, \\
\text{Categorical}(\{-s, -t\}), & i=7,
\end{cases}
\end{equation}
\begin{equation}
drive \sim \text{Uniform}(0, 30)
\end{equation}
\begin{equation}
stretch \sim \text{Uniform}(0.9, 1.1)
\end{equation}
\begin{equation}
speed \sim \text{Uniform}(0.9, 1.1)
\end{equation}
\begin{equation}
tremolo\_speed \sim \text{Uniform}(0.1, 100)
\end{equation}
\begin{equation}
tremolo\_depth \sim \text{Uniform}(1, 101)
\end{equation}
\begin{equation}
snr\_range \sim \text{Uniform}(12, 100)
\end{equation}

Let $N$ be the noise matrix with dimensions [channels, time domain samples] as $W$ for white noise. Calculate signal power $P_s$ and noise power $P_n$:

\begin{equation}
P_s = \sum (W_{ij})^2
\end{equation}
\begin{equation}
P_n = \sum (N_{ij})^2
\end{equation}

Calculate the scaling factor $\alpha$ for noise based on the desired signal-to-noise ratio (SNR):

\begin{equation}
\alpha = \sqrt{\frac{P_s}{P_n \times 10^{\frac{snr\_range}{10}}}}
\end{equation}

Compute the scaled noise matrix $N_s$ and the noisy waveform matrix $W_n$:

\begin{equation}
N_s = \alpha \times N
\end{equation}
\begin{equation}
W_n = W + N_s
\end{equation}

The final positive waveform $W_p$ is obtained with noise added according to the chosen SNR. This pipeline enhances the diversity of the positive samples and aims to improve the model's ability to retrieve content-based similar-sounding samples sharing the same underlying musical language.

\subsubsection{Negative sample generation}

Let $a$ be the original anchor clip and $n$ be the negative clip generated. The steps for generating the negative clip $n$ are as follows:

\begin{enumerate}
\item Calculate the length $L_a$ and duration $T_a$ of the anchor clip $a$:

\begin{equation}
L_a = \text{len}(a)
\end{equation}

\begin{equation}
T_a = \frac{L_a}{f_s}
\end{equation}

where $\text{len}(a)$ returns the shape of the audio clip, and $f_s$ is the sample rate of the audio.

\item Determine the number of chunks $n_c$ to split the anchor clip into based on a minimum chunk duration $T_c$:

\begin{equation}
n_c = \left\lfloor\frac{T_a}{T_c}\right\rfloor
\end{equation}

\item Calculate the minimum and maximum chunk lengths in samples:

\begin{equation}
L_{c,\text{min}} = T_c \times f_s
\end{equation}
\begin{equation}
L_{c,\text{max}} = T_{\text{max}} \times f_s
\end{equation}

where $T_{\text{max}}$ is the maximum chunk duration specified in seconds.

\item Generate $n_c$ chunk lengths $L_{c,1},\ldots,L_{c,n_c}$ as follows:

\begin{equation}
L_{c,i} = \begin{cases}
L_{c,\text{min}}, & i=1, \\
\text{randint}(L_{c,\text{min}}, L_{c,\text{max}}), & 1<i<n_c, \\
L_a - \sum_{j=1}^{i-1} L_{c,j}, & i=n_c.
\end{cases}
\end{equation}

\item Split the anchor clip into chunks $c_1,\ldots,c_{n_c}$ based on the lengths $L_{c,1},\ldots,L_{c,n_c}$:

\begin{equation}
c_i = a[..., s_i : s_i + L_{c,i}]
\end{equation}
\begin{equation}
S_c = [0] + [L_{c,1},...,L_{c,n_c-1}]
S_c = \sum_{i=0}^{n_c-1} S_{c,i}
\end{equation}
\begin{equation}
s_i = S_c[i]
\end{equation}
\begin{equation}
i=1,\ldots,n_c
\end{equation}

\item Shuffle the resulting list of chunks randomly to obtain the reordered chips $c'_1,\ldots,c'_{n_c}$:

Let $p$ be a random permutation of the indices $1$ to $n_c$. Then, the shuffled array $c'$ is given by:
\begin{equation}
c' = [c_{p(1)}, c_{p(2)}, \dots, c_{p(n_c)}]
\end{equation}


\item Let $c'_1, \dots, c'_{n_c}$ be the shuffled chunks. Then, the negative clip $n$ is obtained by concatenating the shuffled chunks:
\begin{equation}
n = [c'_1, \dots, c'_{n_c}]
\end{equation}


\end{enumerate}

\subsection{Loss function}

The mathematical expression for the triplet loss function with a variable margin can be written as follows:

\begin{equation}
L(\mathbf{a}, \mathbf{p}, \mathbf{n}) = \frac{1}{N} \sum_{i=1}^{N} \max \left(0, m + \left| \mathbf{a}_i - \mathbf{p}_i \right|_2^2 - \left| \mathbf{a}_i - \mathbf{n}_i \right|_2^2 \right)
\end{equation}

In the triplet loss function, $d(\mathbf{a}, \mathbf{p})$ and $d(\mathbf{a}, \mathbf{n})$ denote the Euclidean distance (L2-norm) between the anchor $\mathbf{a}$ and the positive sample $\mathbf{p}$ and the negative sample $\mathbf{n}$, respectively. The margin $m$ is a positive constant determining the minimum distance between the anchor-positive and anchor-negative pairs.

The $\max(0, ...)$ function computes the maximum of the expression within the parentheses and 0. This function aims to ensure that the loss is only calculated for triplets that violate the margin constraint, where the negative sample is closer to the anchor than the positive sample by at least the margin. When the distance between the positive and negative samples is greater than the margin, the loss is zero. Otherwise, the loss is proportional to the extent to which the constraint is violated.

The final form of the loss function is obtained by averaging the loss over a mini-batch of triplets. 

\begin{equation}
L = \frac{1}{N} \sum_{i=1}^{N} L(anchor_i, positive_i, negative_i)
\end{equation}

Here, $N$ is the number of triplets in the mini-batch, and $L(anchor_i, positive_i, negative_i)$ represents the loss function computed for the i-th triplet in the batch.