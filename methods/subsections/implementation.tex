\section{Implementation details}

Our approach falls broadly under cognitive modeling, which seeks to simulate human cognitive processes and problem-solving through a computerized model. We advocate for exposure-based learning in music, encouraging active engagement with various musical styles, genres, techniques, and learning methods. This approach promotes comprehensive musical proficiency and efficient performance when encountering novel data in practical applications.

According to Piaget's theory of cognitive development, children gain knowledge through sensory experiences and gradually develop abstract reasoning and schemas—basic cognitive structures \cite{Huitt2003PiagetsDevelopment}. These schemas evolve by incorporating new information through processes of assimilation and accommodation \cite{audioselfsupsurvey}.

In pattern recognition, models are designed to exhibit robustness against known invariances—transformations of input data—thereby ensuring consistent output. Even unknown invariances not explicitly considered in the model's design can be accommodated due to the model's inherent learning capacity.

Our implementation offers a slight modification to the \textit{contrastive learning of musical representations} (CLMR), a method that learns valuable, discriminative music representations without explicit labels by contrasting positive augmentations of a musical piece against negative ones \cite{CLMR2021}. CLMR falls under a branch of ML known as self-supervised learning (SSL) \cite{Balestriero2023ALearning}, where models learn autonomously from unlabeled data to create their own supervisory signals \cite{audioselfsupsurvey}. This method resembles how humans learn from observations and interactions, transforming unsupervised problems into supervised ones by auto-generating labels. SSL benefits include reduced dependence on labeled data, contributing to more robust and generalizable data representations.

As previously stated, the methodology of CLMR involves generating different augmentations of the same musical piece to serve as positive pairs while using other pieces as negative pairs. The training process encourages the model to produce similar representations for positive and dissimilar ones for negative pairs.

Although SSL has proven effective in speech and audio, its application to music audio remains relatively unexplored. This is primarily due to the unique challenges of modeling musical knowledge, especially regarding music's tonal and pitched characteristics \cite{Li2023MERT:Training}.

\subsection{Deep architecture design}

This work suggests employing a Triple Siamese Network (TSN), a model architecture known for its efficacy in music similarity retrieval tasks \cite{contentmusicsimtriplet2020}. The aim is to minimize the loss function between a triplet of anchor, positive, and negative samples, utilizing online triplet mining for optimizing memory resources \cite{Sikaroudi2020OfflinePatches}. This SSL model aims to distinguish between similar and dissimilar samples effectively.

Introduced by Bromley and LeCun \cite{Bromley1993SignatureNetwork}, Siamese Networks are deep learning architectures designed for tasks requiring comparison or similarity assessment between instances. The architecture comprises identical subnetworks that share the same parameters, improving memory usage and computational efficiency. Rather than learning specific features of individual classes, they focus on a similarity metric, making them ideal for imbalanced datasets. Each subnetwork processes an input independently, combining the outputs to yield a similarity score. Training with shared weights enables the model to learn invariant input representation, improving comparison efficiency. This is accomplished through a specialized loss function called triplet loss, which aims to minimize the distance between similar inputs and maximize it for dissimilar ones.

The TSN extends the binary contrastive architecture by comparing three input instances instead of two. It strives to learn an embedding space where similar instances are closer and dissimilar ones are more distant. The inputs, consisting of an anchor, a positive instance (similar to the anchor), and a negative instance (dissimilar to the anchor), are processed by identical subnetworks sharing the same parameters.

\subsubsection{Encoder architecture: SampleCNN}

The SampleCNN model \cite{Lee2018SampleCNN:Classification} is a CNN designed for raw waveform audio data, treating each audio sample as an independent channel and applying 1-dimensional convolution along the temporal axis. Its implementation is an adaptation of \cite{CLMR2021} using \textit{PyTorch} \cite{Paszke2019PyTorch:Library} and \textit{PyTorch Lightning} \cite{PyTorchDocumentation}.

With only 2.4 million trainable parameters, this fully convolutional model reduces computational requirements and learns features at different scales through its multi-resolution architecture \ref{tab:samplecnntable}.

\begin{table}[ht]
\centering
\small
\begin{tabularx}{\textwidth}{>{\hsize=1.6\hsize}X>{\hsize=0.6\hsize}X>{\hsize=0.6\hsize}X}
\toprule
\thead{\textbf{Layer Type}} & \thead{\textbf{In Channels}} & \thead{\textbf{Out Channels}} \\
\midrule
Conv + ReLU & 1 & 128 \\
\addlinespace
\multicolumn{3}{c}{The following layers are repeated depending on the strides and hidden parameters:} \\
\addlinespace
Conv + BatchNorm + ReLU + MaxPool & 128 & 128 \\
\addlinespace
Conv + BatchNorm + ReLU + MaxPool & 128 & 128 \\
\addlinespace
Conv + BatchNorm + ReLU + MaxPool & 128 & 256 \\
\addlinespace
Conv + BatchNorm + ReLU + MaxPool & 256 & 256 \\
\addlinespace
Conv + BatchNorm + ReLU + MaxPool & 256 & 256 \\
\addlinespace
Conv + BatchNorm + ReLU + MaxPool & 256 & 256 \\
\addlinespace
Conv + BatchNorm + ReLU + MaxPool & 256 & 256 \\
\addlinespace
Conv + BatchNorm + ReLU + MaxPool & 256 & 256 \\
\addlinespace
Conv + BatchNorm + ReLU + MaxPool & 256 & 512 \\
\addlinespace
Conv + BatchNorm + ReLU + AvgPool & 512 & 512 \\
\addlinespace
Fully connected or representation layer & 512 & 128 \\
\bottomrule
\end{tabularx}
\caption{Layer specifications for SampleCNN model}
\label{tab:samplecnntable}
\end{table}

The original model has been subtly adjusted by introducing an average pooling operation to the final convolutional layer. This modification, dimensionality reduction, is strategic for handling various input data sizes. Condensing the feature maps to a fixed-size output ensures consistency for subsequent layers like fully connected ones, streamlining computations while maintaining crucial information. This results in a more robust, efficient learning process that offers translation invariance and better contextual understanding.

%Colour CNN
\begin{figure}[ht]
    \centering
% DEEP CONVOLUTIONAL NEURAL NETWORK
\begin{tikzpicture}[x=1.6cm,y=1.1cm]
  \large
  \message{^^JDeep convolution neural network}
  \readlist\Nnod{5,5,4,3,2,4,4,3} % array of number of nodes per layer
  \def\NC{6} % number of convolutional layers
  \def\nstyle{int(\lay<\Nnodlen?(\lay<\NC?min(2,\lay):3):4)} % map layer number on 1, 2, or 3
  \tikzset{ % node styles, numbered for easy mapping with \nstyle
    node 1/.style={node in},
    node 2/.style={node convol},
    node 3/.style={node hidden},
    node 4/.style={node out},
  }
  
  % TRAPEZIA
  \draw[myorange!40,fill=myorange,fill opacity=0.02,rounded corners=2]
    %(1.6,-2.5) rectangle (4.4,2.5);
    (1.6,-2.7) --++ (0,5.4) --++ (3.8,-1.9) --++ (0,-1.6) -- cycle;
  \draw[myblue!40,fill=myblue,fill opacity=0.02,rounded corners=2]
    (5.6,-2.0) rectangle++ (1.8,4.0);
  \node[right=19,above=1,align=center,myorange!60!black] at (3.1,1.8) {convolutional\\[-0.2em]layers};
  \node[above=1,align=center,myblue!60!black] at (6.5,1.9) {fully-connected\\[-0.2em]hidden layers};
  
  \message{^^J  Layer}
  \foreachitem \N \in \Nnod{ % loop over layers
    \def\lay{\Ncnt} % alias of index of current layer
    \pgfmathsetmacro\prev{int(\Ncnt-1)} % number of previous layer
    %\pgfmathsetmacro\Nprev{\Nnod[\prev]} % array of number of nodes in previous layer
    \message{\lay,}
    \foreach \i [evaluate={\y=\N/2-\i+0.5; \x=\lay; \n=\nstyle;}] in {1,...,\N}{ % loop over nodes
      %\message{^^J  Layer \lay, node \i}
      
      % NODES
      \node[node \n,outer sep=0.6] (N\lay-\i) at (\x,\y) {};
      
      % CONNECTIONS
      \ifnum\lay>1 % connect to previous layer
        \ifnum\lay<\NC % convolutional layers
          \foreach \j [evaluate={\jprev=int(\i-\j); \cconv=int(\Nnod[\prev]>\N); \ctwo=(\cconv&&\j>0);
                       \c=int((\jprev<1||\jprev>\Nnod[\prev]||\ctwo)?0:1);}]
                       in {-1,0,1}{
            \ifnum\c=1
              \ifnum\cconv=0
                \draw[connect,white,line width=1.2] (N\prev-\jprev) -- (N\lay-\i);
              \fi
              \draw[connect] (N\prev-\jprev) -- (N\lay-\i);
            \fi
          }
          
        \else % fully connected layers
          \foreach \j in {1,...,\Nnod[\prev]}{ % loop over nodes in previous layer
            \draw[connect,white,line width=1.2] (N\prev-\j) -- (N\lay-\i);
            \draw[connect] (N\prev-\j) -- (N\lay-\i);
          }
        \fi
      \fi % else: nothing to connect first layer
      
    }
  }
  
  % LABELS
  \node[above=0.5,align=center,mygreen!60!black] at (N1-1.90) {input\\[-0.2em]layer};
  \node[above=1.5,align=center,myred!60!black] at (N\Nnodlen-1.90) {output\\[-0.2em]layer};
  
\end{tikzpicture}
    \caption{High-level illustration of a CNN}
    \label{fig: CNN colour diagram}
\end{figure}

\subsection{Optimizer and learning rate}

We've employed the AdamW optimizer \cite{Loshchilov2017DecoupledRegularization}, an optimized variant of the widely-used Adam optimizer for training neural networks. AdamW adeptly balances the learning rate across network weights, providing an efficient strategy for weight decay management by isolating it from gradient updates. 

The learning rate, set at 0.003, is a pivotal parameter dictating the step size at each iteration towards loss function minimization. It's a delicate balancing act— a high rate promises swift convergence with a risk of minimum overshoot, while a lower rate provides careful convergence but necessitates more iterations. Given that delicacy, we set it to a standard number broadly used in the literature.

\subsection{Audio augmentation and transformation pipeline}

The choice of input data is guided by the task, computational resources, and the need to balance data retention with computational efficiency.

Previous research has utilized CNNs with various features such as Mel-Scaled Log-magnitude Spectograms (MLS), Self-Similarity Matrices (SSM), and Self-Similarity Lag Matrices (SSLM) as inputs \cite{Hernandez-Olivan2021MusicFeatures}. However, features derived from raw audio may lack interpretability in some scenarios \cite{Schindler2020DeepTutorial}, and raw audio presents unique advantages despite being highly computationally demanding. It ensures the preservation of the original signal, potentially uncovering novel insights, and allows for direct feature extraction via advanced DL models \cite{learning, verydeep}. Nevertheless, it comes with challenges, such as high dimensionality requiring substantial computational resources. 

Time-domain processing naturally handles temporal patterns and sequences in data, thereby avoiding windowing artifacts. Although audio feature-based methods are effective for various audio-related machine learning tasks, their limitations lie in representing perceptual similarity. Mel-Scaled Log-magnitude Spectograms, for example, capture frequency distribution over time. Yet, the complexity of human auditory perception—encompassing temporal patterns, phase relationships between frequencies, and higher-level musical structures means that musically similar sounds can have distinct spectrograms. This discrepancy implies that using spectrogram distance alone for measuring high-level music content may not always align with human perceptions \cite{Kim2020OneStrategies, Mesostructures2023}.

\subsubsection{Positive sample generation chain}

The positive image in every triplet of input data must preserve its intelligible content when subjected to transformations, regardless of alterations in sonic qualities and processing artifacts. Maintaining the temporal structure and meaningfulness of the content allows it to present musical elements remarkably close to the original track.

While we have experimented with helpful audio augmentation tools such \cite{Spijkervet2021Spijkervet/torchaudio-augmentations:V1.0, Kharitonov2020DataDomain}, the specific requirements of our experiments required the development of our own transformation chain using \textit{torchaudio}'s \cite{Yang2021TorchAudio:Processing} implementation of \textit{SoX} \cite{sox}: given an anchor audio signal $A[n]$, we generate a positive signal $P[n]$ by applying a series of amplitude, time-domain, frequency-domain, modulation, reverberation, and nonlinear effects with additive noise on top of it. 

\textbf{Amplitude effects}: The signal's amplitude is modified by a constant factor using gain $g \in [-12, 0]$.

\textbf{Time-domain effects}: The signal's playback speed and duration are altered through speed change and stretching, preserving the relative perceptual musical relationships between wave points. The respective factors are $\alpha \in [0.9, 1.1]$ for speed change and $\beta \in [0.9, 1.1]$ for stretch.

\textbf{Frequency-domain effects}: The frequency content is adjusted through pitch-shifting, modifying the pitch by $\Delta p \in [-1200, 1200]$ cents.

\textbf{Nonlinear effects}: Nonlinear distortion is introduced via overdrive with a parameter $d \in [0, 30]$.

\textbf{Modulation effects}: Utilize a control signal or low-frequency oscillator. The chorus parameters are determined by a set of six different variables: $f_d, f_m, f_b, f_w, f_a, f_s$ where $f_d, f_m, f_w, f_a$ are floats in the range $[0.1, 1.0]$, $f_b$ is an integer in the range $[20, 55]$, $f_s$ is an integer in the range $[2, 5]$ and the type of control signal is chosen between "s" and "t". The tremolo's amplitude modulation frequency and depth are controlled by $t_s \in [0.1, 100]$ and $t_d \in [1, 101]$, respectively.

\textbf{Reverberation effects} simulate a physical space's acoustic reflections and reverberations by applying an impulse response $h_R[n]$.

\textbf{Noise effects}: A noise signal $Noise[n]$ is added with a signal-to-noise ratio (SNR) in the range $[12, 100]$.

The positive signal $P[n]$ is generated according to the expression given below, where the symbols denote specific effects as indicated:

\begin{equation}\label{eq:positive_signal}
P[n] = A[n] \ast h_{G}[n] \ast h_{C}[n] \ast h_{D}[n] \ast h_{P_t}[n] \ast h_{S}[n] \ast h_{T}[n] + N_{SNR}[n]
\end{equation}

The $\ast$ symbol represents convolution, and the various $h$ symbols indicate the impulse responses of specific effects. The noise signals $N_{SNR}[n]$ are included with a specific SNR.

Random parameter updates within hardcoded ranges generate unique audio $P[n]$ out of $A[n]$ for each run-through. Adding random white noise and varying SNR creates countless noisy waveform variations. Although possible combinations can be estimated by multiplying discrete parameter values, the presence of continuous parameters and randomness in noise generation effectively results in infinite unique audio versions.

\subsubsection{Negative sample generation}

Circling back to our premises and assumptions, we posit that high-level musical content unfolds over time; therefore, we argue that the temporal structure of the negative images, in contrast to our anchor, should be disrupted. While maintaining similar sonic qualities, the content should be rendered unintelligible.

The computation for the negative signal $N[n]$ per every $A[n]$ goes as follows:

\begin{enumerate}
\item We first calculate the minimum and maximum audio chunk lengths in samples:
\begin{align}
l_{min} &= t_{min} \times S, &
l_{max} &= t_{max} \times S
\end{align}

The minimum duration $t_{min}$ is set to 0.05 seconds, and the maximum duration $t_{max}$ is set to 1 second. This range is chosen thoughtfully to strike a balance between two factors: on the one hand, it is above the just noticeable difference (JND) threshold, the smallest change in a stimulus that can be perceived. On the other hand, it is short enough to maintain a reasonably-sized window to avoid discernible musical content \cite{Fastl2007Just-NoticeableChanges}.

\item We then generate random audio chunk lengths $l_1, l_2, \ldots, l_{n-1}$ from the uniform distribution on the interval $[l_{min}, l_{max}]$. Calculate the final audio chunk length as:
\begin{equation}
l_n = L_A - \sum_{i=1}^{n-1} l_i
\end{equation}
where $L_A$ is the length of the anchor signal in samples.

\item The third step is to split the anchor signal $A$ into audio chunks $C_1, C_2, \ldots, C_n$ according to the calculated audio chunk lengths in the previous step.

\item Shuffle the audio chunks randomly to get the permuted slices $C_{\sigma(1)}, C_{\sigma(2)}, \ldots, C_{\sigma(n)}$, where $\sigma$ is a random permutation of indices from $1$ to $n$. 

\item We finally concatenate the shuffled audio chunks to generate the negative signal that will have similar production while the content is completely ruined:
\begin{equation}\label{eq:negative_signal}
N[n] = C_{\sigma(1)} \oplus C_{\sigma(2)} \oplus \ldots \oplus C_{\sigma(n)}
\end{equation}
\end{enumerate}

The whole purpose of this process is to disturb the content unfolding in the time domain so it becomes musically unintelligible while maintaining the production and sonic attributes.

\subsection{Loss function}

Schroff, F., Kalenichenko, D., and Philbin, J. from Google first proposed and applied triplet loss for the learning of facial recognition, catering to varied poses and angles of the same individual \cite{Schroff2015FaceNet:Clustering}.

On the contrary to the also popular contrastive loss \cite{supercontrast}, the triplet loss function directs the learning process by minimizing the distance between the anchor and positive instances \textbf{and} maximizing the distance between the anchor and negative instances. Including a margin parameter in the loss function guarantees a minimum separation between positive and negative instances in the embedding space.

The triplet loss function $\mathcal{L}(\mathbf{a}, \mathbf{p}, \mathbf{n})$ aims to ensure that an anchor vector $\mathbf{a}_i$ is closer in the embedding space to a positive vector $\mathbf{p}_i$ (representing an example of the same class) than to a negative vector $\mathbf{n}_i$ (representing an example of a different class) by at least a margin $\alpha$. It is calculated by summing the losses overall $N$ triplets in the dataset, where the equation gives the loss for each triplet:

\begin{equation}
\mathcal{L}(\mathbf{a}, \mathbf{p}, \mathbf{n}) = \sum_{i=1}^{N} \max \left(0, \left| \mathbf{a}_i - \mathbf{p}_i \right|_2^2 - \left| \mathbf{a}_i - \mathbf{n}_i \right|_2^2 + \alpha \right)
\end{equation}

The $\max(0, x)$ operation ensures zero loss when the distances satisfy this condition. The final loss used for model training is then the average loss over a mini-batch of $N$ triplets:

\begin{equation}
\mathcal{L} = \frac{1}{N} \sum_{i=1}^{N} \mathcal{L}(\mathbf{a}_i, \mathbf{p}_i, \mathbf{n}_i)
\end{equation}

The margin is a task-dependent optimal value determined empirically based on model performance. If it's too small, the model might not differentiate classes effectively; if it's too large, it might focus on outliers.

While some packages can be found in the MIR literature \cite{auraloss}, we wrote our own PyTorch \cite{Paszke2019PyTorch:Library} implementation for the sake of our experiments.

As previously stated, the goal of minimizing this loss function is to learn discriminative embeddings, where similar examples are grouped closely together. In contrast, dissimilar examples are placed farther apart in the embedding space.

\subsection{Online triplet mining and batch normalization}

Online triplet mining is beneficial for managing large datasets by dynamically selecting the most informative triplets during training, focusing on each mini-batch. This strategy makes the process memory-efficient by negating the need to store all possible triplet combinations. Still, it also enhances model performance through quicker convergence by focusing on challenging examples based on the current model state.

This hard triplet mining selects triplets $(a, p, n)$ to maximize the Euclidean distance between the anchor and positive samples and the anchor and negative samples. These distances, $D_{\text{AP}}$ and $D_{\text{AN}}$, are computed respectively as:


\begin{align}
D_{\text{AP}} &= \sqrt{\sum_{i} (A_i - P_i)^2} & D_{\text{AN}} &= \sqrt{\sum_{i} (A_i - N_i)^2}
\end{align}


In implementing the batch normalization step, it is necessary to standardize the audio lengths across all elements in the minibatch. We opted to zero-pad all clips to the length of the longest clip, valuing data integrity and completeness over potential performance trade-offs. Thus, the length of the longest array in the batch, which sets the standard for all others, is as follows:

\begin{equation}
L_{\text{max}} = \max_{i \in I} \left( \max \left( |A_i|, |P_i|, |N_i| \right) \right)
\end{equation}

$I$ represents the set of all items in the batch, $|A_i|$, $|P_i|$, and $|N_i|$ denote the lengths of the anchor, positive, and negative vectors for the $i$-th item, respectively. The $\max$ function is applied to find the longest of these three lengths for each item, and then the maximum of these maximum lengths is taken over all items in the batch. This gives the maximum length, $L_{\text{max}}$, of any vector in the batch.

\subsubsection{Hardware and training strategy}

The deep learning models were trained on a high-performance cloud computing setup hosted on \href{https://cloud.google.com/}{Google Cloud Platform}. The machine was of type \texttt{n1-standard-32}, equipped with an Intel Skylake processor and four NVIDIA Tesla T4 GPUs. The multiple GPUs allowed for efficient parallel processing, significantly reducing the training time.

To tackle the computational demands of training models on extensive raw audio data, we incorporated a couple of strategies to optimize efficiency and performance. 

First, we utilized 16-mixed precision training. This approach leverages the improved performance of modern GPUs for 16-bit computations, enabling the model to run faster and use less memory without sacrificing model performance \cite{Das2018MixedOperations}.

Secondly, to capitalize on the computational capabilities of multiple GPUs and hasten training times, we employed the Distributed Data-Parallel (DDP) strategy \cite{Li2020PyTorchTraining}. DDP operates on distinct mini-batches of data across GPUs and synchronizes the gradients after each backward pass, providing a more efficient scaling than other parallel strategies. 

These strategies collectively enhance the computational efficiency while maintaining the robustness of the model training on lengthy raw audio data.

\newpage