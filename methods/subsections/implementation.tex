\section{Implementation details}

Our approach falls broadly under cognitive modeling, which seeks to simulate human cognitive processes and problem-solving through a computerized model. We advocate for exposure-based learning in music, encouraging active engagement with various musical styles, genres, techniques, and learning methods. This approach promotes comprehensive musical proficiency and efficient performance when encountering novel data in practical applications.

According to Piaget's theory of cognitive development, children gain knowledge through sensory experiences and gradually develop abstract reasoning and schemas—basic cognitive structures \cite{Huitt2003PiagetsDevelopment}. These schemas evolve by incorporating new information through processes of assimilation and accommodation \cite{audioselfsupsurvey}.

In pattern recognition, models are designed to exhibit robustness against known invariances—transformations of input data—thereby ensuring consistent output. Interestingly, even unknown invariances not explicitly considered in the model's design can be accommodated due to the model's inherent learning capacity.

Contrastive Learning of Musical Representations (CLMR) is a method that learns valuable, discriminative music representations without explicit labels by contrasting positive augmentations of a musical piece against negative ones \cite{CLMR2021}. CLMR falls under a subset of machine learning called self-supervised learning (SSL) \cite{Balestriero2023ALearning}, where models learn independently from unlabeled data to create their supervisory signals \cite{audioselfsupsurvey}. This method resembles how humans learn from observations and interactions, turning unsupervised problems into supervised ones by auto-generating labels. Benefits of SSL include reduced dependence on labeled data, which contributes to more robust and generalizable data representations.

The methodology of CLMR involves generating different augmentations of the same musical piece to serve as positive pairs while using other pieces as antagonistic pairs. The training process, which minimizes a contrastive loss function, encourages the model to produce similar representations for positive and dissimilar ones for antagonistic pairs.

Although SSL has proven effective in speech and audio, its application to music audio remains relatively unexplored. This is primarily due to the unique challenges associated with modeling musical knowledge, especially when it comes to music's tonal and pitched characteristics \cite{Li2023MERT:Training}.

\subsection{Triplet Siamese Networks}

For this task, a Triple Siamese Network is suggested, a model architecture proved efficient for music similarity retrieval tasks \cite{contentmusicsimtriplet2020}, intending to minimize the loss function between an anchor, positive, and negative sample, achieved through online triplet mining \cite{Sikaroudi2020OfflinePatches}. The overarching aim is to train a model in a self-supervised setting, which can effectively distinguish between similar and dissimilar samples. 

A Siamese Network is a deep learning architecture introduced by Bromley and LeCun in the early 1990s \cite{Bromley1993SignatureNetwork}. This architecture is explicitly designed for tasks requiring comparing or assessing similarity between two input instances. It comprises two or more identical subnetworks connected in parallel and joined at the output layer. These subnetworks share the same architecture, weights, and hyperparameters, allowing more efficient memory usage and computational complexity.

Siamese Networks are especially effective for learning from limited or imbalanced datasets, as they focus on learning a similarity metric rather than the specific features of individual classes. Each subnetwork processes an input independently, and its outputs are then combined and further processed to yield a single similarity score. The shared weights during training enable the model to learn an invariant representation for the input instances, thereby enhancing its efficiency in comparing and contrasting them. To achieve this, Siamese Networks employ specialized loss functions, such as contrastive or triplet loss, which aim to minimize the distance between similar input pairs and maximize the distance between dissimilar ones.

An extension of this architecture is the Triplet Siamese Network, which involves comparing three input instances instead of two. This network aims to learn an embedding space in which similar instances are close to each other and dissimilar instances are distant. The input to this network consists of an anchor, a positive instance (from the same class as the anchor), and a negative instance (from a different class). Each instance is processed by an identical subnetwork that shares its architecture, weights, and hyperparameters with the others.

\subsubsection{Encoder: SampleCNN}

The SampleCNN model \cite{Lee2018SampleCNN:Classification} is a CNN designed for raw waveform audio data, treating each audio sample as an independent channel and applying 1-dimensional convolution along the temporal axis. Its implementation is an adaptation of \cite{CLMR2021} using PyTorch \cite{Paszke2019PyTorch:Library} and PyTorch Lightning \cite{PyTorchDocumentation}.

This fully convolutional model reduces computational requirements and learns features at different scales through its multi-resolution architecture \ref{tab:samplecnn}.

\begin{table}[h]
\centering
\small
\begin{tabularx}{\textwidth}{>{\hsize=1.6\hsize}X>{\hsize=0.6\hsize}X>{\hsize=0.6\hsize}X}
\toprule
\thead{\textbf{Layer Type}} & \thead{\textbf{In Channels}} & \thead{\textbf{Out Channels}} \\
\midrule
Conv + ReLU & 1 & 128 \\
\addlinespace
\multicolumn{3}{c}{The following layers are repeated depending on the strides and hidden parameters:} \\
\addlinespace
Conv + BatchNorm + ReLU + MaxPool & 128 & 128 \\
\addlinespace
Conv + BatchNorm + ReLU + MaxPool & 128 & 128 \\
\addlinespace
Conv + BatchNorm + ReLU + MaxPool & 128 & 256 \\
\addlinespace
Conv + BatchNorm + ReLU + MaxPool & 256 & 256 \\
\addlinespace
Conv + BatchNorm + ReLU + MaxPool & 256 & 256 \\
\addlinespace
Conv + BatchNorm + ReLU + MaxPool & 256 & 256 \\
\addlinespace
Conv + BatchNorm + ReLU + MaxPool & 256 & 256 \\
\addlinespace
Conv + BatchNorm + ReLU + MaxPool & 256 & 256 \\
\addlinespace
Conv + BatchNorm + ReLU + MaxPool & 256 & 512 \\
\addlinespace
Conv + BatchNorm + ReLU + AvgPool & 512 & 512 \\
\addlinespace
Fully connected (representation layer) & 512 & 128 \\
\bottomrule
\end{tabularx}
\caption{Layer specifications for SampleCNN model}
\label{tab:samplecnn}
\end{table}


The original model has been modified slightly to serve our specific needs: the output from the final convolutional layer is now subjected to an average pooling operation at the end.

%Colour CNN
\begin{figure}[ht]
    \centering
% DEEP CONVOLUTIONAL NEURAL NETWORK
\begin{tikzpicture}[x=1.6cm,y=1.1cm]
  \large
  \message{^^JDeep convolution neural network}
  \readlist\Nnod{5,5,4,3,2,4,4,3} % array of number of nodes per layer
  \def\NC{6} % number of convolutional layers
  \def\nstyle{int(\lay<\Nnodlen?(\lay<\NC?min(2,\lay):3):4)} % map layer number on 1, 2, or 3
  \tikzset{ % node styles, numbered for easy mapping with \nstyle
    node 1/.style={node in},
    node 2/.style={node convol},
    node 3/.style={node hidden},
    node 4/.style={node out},
  }
  
  % TRAPEZIA
  \draw[myorange!40,fill=myorange,fill opacity=0.02,rounded corners=2]
    %(1.6,-2.5) rectangle (4.4,2.5);
    (1.6,-2.7) --++ (0,5.4) --++ (3.8,-1.9) --++ (0,-1.6) -- cycle;
  \draw[myblue!40,fill=myblue,fill opacity=0.02,rounded corners=2]
    (5.6,-2.0) rectangle++ (1.8,4.0);
  \node[right=19,above=1,align=center,myorange!60!black] at (3.1,1.8) {convolutional\\[-0.2em]layers};
  \node[above=1,align=center,myblue!60!black] at (6.5,1.9) {fully-connected\\[-0.2em]hidden layers};
  
  \message{^^J  Layer}
  \foreachitem \N \in \Nnod{ % loop over layers
    \def\lay{\Ncnt} % alias of index of current layer
    \pgfmathsetmacro\prev{int(\Ncnt-1)} % number of previous layer
    %\pgfmathsetmacro\Nprev{\Nnod[\prev]} % array of number of nodes in previous layer
    \message{\lay,}
    \foreach \i [evaluate={\y=\N/2-\i+0.5; \x=\lay; \n=\nstyle;}] in {1,...,\N}{ % loop over nodes
      %\message{^^J  Layer \lay, node \i}
      
      % NODES
      \node[node \n,outer sep=0.6] (N\lay-\i) at (\x,\y) {};
      
      % CONNECTIONS
      \ifnum\lay>1 % connect to previous layer
        \ifnum\lay<\NC % convolutional layers
          \foreach \j [evaluate={\jprev=int(\i-\j); \cconv=int(\Nnod[\prev]>\N); \ctwo=(\cconv&&\j>0);
                       \c=int((\jprev<1||\jprev>\Nnod[\prev]||\ctwo)?0:1);}]
                       in {-1,0,1}{
            \ifnum\c=1
              \ifnum\cconv=0
                \draw[connect,white,line width=1.2] (N\prev-\jprev) -- (N\lay-\i);
              \fi
              \draw[connect] (N\prev-\jprev) -- (N\lay-\i);
            \fi
          }
          
        \else % fully connected layers
          \foreach \j in {1,...,\Nnod[\prev]}{ % loop over nodes in previous layer
            \draw[connect,white,line width=1.2] (N\prev-\j) -- (N\lay-\i);
            \draw[connect] (N\prev-\j) -- (N\lay-\i);
          }
        \fi
      \fi % else: nothing to connect first layer
      
    }
  }
  
  % LABELS
  \node[above=0.5,align=center,mygreen!60!black] at (N1-1.90) {input\\[-0.2em]layer};
  \node[above=1.5,align=center,myred!60!black] at (N\Nnodlen-1.90) {output\\[-0.2em]layer};
  
\end{tikzpicture}
    \caption{High-level illustration of a CNN}
    \label{fig: CNN colour diagram}
\end{figure}

\subsection{Optimizer and learning rate}

In our research, we've employed the AdamW optimizer \cite{Loshchilov2017DecoupledRegularization}, an optimized variant of the widely-used Adam optimizer for training neural networks. AdamW adeptly balances the learning rate across network weights, providing an efficient strategy for weight decay management by isolating it from gradient updates. 

The learning rate, set at 0.003, is a pivotal parameter dictating the step size at each iteration towards loss function minimization. It's a delicate balancing act— a high rate promises swift convergence with a risk of minimum overshoot, while a lower rate provides careful convergence but necessitates more iterations.


\subsection{Audio augmentation and transformation pipeline}

Features derived from raw audio often lack the interpretability of traditional ones \cite{Schindler2020DeepTutorial}. Hence, the choice of input data depends on the specific task, available computational resources, and the balance between data preservation and computational efficiency.

While previous studies have used CNNs and various features as input like Mel-Scaled Log-magnitude Spectograms (MLS), Self-Similarity Matrices (SSM), and Self-Similarity Lag Matrices (SSLM) as inputs \cite{Hernandez-Olivan2021MusicFeatures}, raw audio data offers unique advantages: it ensures the retention of the original signal, which could reveal novel findings, and facilitates the direct extraction of features through advanced deep learning models \cite{learning}. However, it is not without challenges: high sample rates result in high dimensionality, demanding substantial computational resources. Additionally, raw audio is noise-sensitive, complicating the extraction process. 

Time-domain processing naturally handles temporal patterns and sequences in data, thereby avoiding windowing artifacts. Although Mel-spectrogram-based methods are effective for various audio-related machine learning tasks, their limitations lie in representing perceptual similarity. Spectrograms capture frequency distribution over time. Yet, the complexity of human auditory perception—encompassing temporal patterns, phase relationships between frequencies, and higher-level musical structures—means that musically similar sounds can have distinct spectrograms. This discrepancy implies that using spectrogram distance alone for measuring high-level music similarity may not always align with human perceptions \cite{Kim2020OneStrategies, Mesostructures2023}.

\subsubsection{Positive sample generation chain}

The positive image in every triplet of input data must preserve its intelligible content when subjected to transformations, regardless of alterations in sonic qualities. Maintaining the temporal structure and meaningfulness of the content allows it to present musical elements identical or strikingly similar to the original track.

While we have experimented with helpful audio augmentation tools \cite{Spijkervet2021Spijkervet/torchaudio-augmentations:V1.0, Kharitonov2020DataDomain}, the specific requirements of our experiments necessitated the development of our transformation chain using \textit{torchaudio}'s \cite{Yang2021TorchAudio:Processing} implementation of \textit{SoX} \cite{sox}: given an anchor audio signal $A[n]$, we generate a positive signal $P[n]$ by applying a series of amplitude, time-domain, frequency-domain, modulation, reverberation, and nonlinear effects with additive noise on top of it. 

\textbf{Amplitude effects}, we use gain to modify the signal's amplitude with a constant factor $g \in [-12, 0]$.

\textbf{Time-domain effects} such as speed change and stretch alter the signal's playback speed and duration without changing its pitch, respectively. The speed and stretch factors are $\alpha \in [0.9, 1.1]$ for speed change and $\beta \in [0.9, 1.1]$ for the stretch factor.

\textbf{Frequency-domain effects} adjust the frequency content or pitch of the signal, as in pitch shifting, which alters the pitch by $\Delta p \in [-1200, 1200]$.

\textbf{Nonlinear effects} such as overdrive add harmonic distortion using the parameter $d \in [0, 30]$ for overdrive

\textbf{Modulation effects}, including chorus and tremolo, use a control signal or low-frequency oscillator. The specified ranges determine the chorus parameters and the tremolo's amplitude modulation frequency and depth $f_m \in [0.1, 100]$ for the tremolo's modulation frequency and $d_m \in [1, 101]$ for the tremolo's depth

\textbf{Reverberation effects} simulate physical space's acoustic reflections and reverberations, like reverb that applies an impulse response $h_R[n]$. 

Finally, \textbf{Noise effects} add a noise signal $N[n]$ with a signal-to-noise ratio (SNR) in the range $[12, 100]$.

The positive signal $P[n]$ is generated according to the equation given below, where the symbols denote specific effects as indicated:

\begin{equation}\label{eq:positive_signal}
P[n] = A[n] \ast h_{G}[n] \ast h_{C}[n] \ast h_{D}[n] \ast h_{P_t}[n] \ast h_{R}[n] \ast h_{S}[n] \ast h_{T}[n] \ast h_{T_m}[n] + N[n]
\end{equation}

The $\ast$ symbol represents convolution, and the various $h$ symbols indicate the impulse responses of specific effects. The noise signal $N[n]$ is included with a specific SNR.

\subsubsection{Negative sample generation}

Circling back to our premises and assumptions, we posit that high-level musical content unfolds over time; therefore, we argue that the temporal structure of the negative images, in contrast to our anchor, should be disrupted. While maintaining similar sonic qualities, the content should be rendered unintelligible.

The computation goes as follows:

\begin{enumerate}
\item We first calculate the minimum and maximum audio chunk lengths in samples:
\begin{equation}
l_{min} = t_{min} \times S
\end{equation}
\begin{equation}
l_{max} = t_{max} \times S
\end{equation}

The minimum duration $t_{min}$ is set to 0.05 seconds, and the maximum duration $t_{max}$ is set to 1 second. This range is chosen thoughtfully to strike a balance between two factors: on the one hand, it is above the just noticeable difference (JND) threshold, the smallest change in a stimulus that can be perceived. On the other hand, it is short enough to maintain a reasonably-sized window to avoid discernible musical content \cite{Fastl2007Just-NoticeableChanges}.

\item We then generate random audio chunk lengths $l_1, l_2, \ldots, l_{n-1}$ from the uniform distribution on the interval $[l_{min}, l_{max}]$. Calculate the final audio chunk length as:
\begin{equation}
l_n = L_A - \sum_{i=1}^{n-1} l_i
\end{equation}
where $L_A$ is the length of the anchor signal in samples.

\item The third step is to split the anchor signal $A$ into audio chunks $C_1, C_2, \ldots, C_n$ according to the calculated audio chunk lengths in the previous step.

\item Shuffle the audio chunks randomly to get the permuted slices $C_{\sigma(1)}, C_{\sigma(2)}, \ldots, C_{\sigma(n)}$, where $\sigma$ is a random permutation of indices from $1$ to $n$. 

\item We finally concatenate the shuffled audio chunks to generate the negative signal that will have similar production while the content is completely ruined:
\begin{equation}\label{eq:negative_signal}
N = C_{\sigma(1)} \oplus C_{\sigma(2)} \oplus \ldots \oplus C_{\sigma(n)}
\end{equation}
\end{enumerate}

The whole purpose of this process is to disturb the content unfolding in the time domain.

\subsection{Loss function}

Schroff, F., Kalenichenko, D., and Philbin, J. at Google initially proposed and used triplet loss to learn face recognition of the same person at different poses and angles. \cite{Schroff2015FaceNet:Clustering}

The triplet loss function used in Siamese-like networks guides the learning process. It seeks to minimize the distance between the anchor and positive instances while maximizing the distance between the anchor and the negative instances. A margin parameter is included in the loss function to ensure a minimum separation between positive and negative instances in the embedding space.

The triplet loss function $\mathcal{L}(\mathbf{a}, \mathbf{p}, \mathbf{n})$ aims to ensure that an anchor vector $\mathbf{a}_i$ is closer in the embedding space to a positive vector $\mathbf{p}_i$ (representing an example of the same class) than to a negative vector $\mathbf{n}_i$ (representing an example of a different class) by at least a margin $m$. It is calculated by summing the losses overall $N$ triplets in the dataset, where the equation gives the loss for each triplet:

\begin{equation}
\mathcal{L}(\mathbf{a}, \mathbf{p}, \mathbf{n}) = \sum_{i=1}^{N} \max \left(0, \left| \mathbf{a}_i - \mathbf{p}_i \right|_2^2 - \left| \mathbf{a}_i - \mathbf{n}_i \right|_2^2 + m \right)
\end{equation}

The final loss used for model training is then the average loss over a mini-batch of $N$ triplets:

\begin{equation}
\mathcal{L} = \frac{1}{N} \sum_{i=1}^{N} \mathcal{L}(\mathbf{a}_i, \mathbf{p}_i, \mathbf{n}_i)
\end{equation}

As previously stated, the goal of minimizing this loss function is to learn discriminative embeddings, where similar examples are grouped closely together. In contrast, dissimilar examples are placed farther apart in the embedding space.

While some packages can be found in the MIR literature \cite{auraloss}, we wrote our own PyTorch \cite{Paszke2019PyTorch:Library} implementation for the sake of our experiments.

\subsection{Online triplet mining and batch normalization}

Online triplet mining is beneficial for managing large datasets by dynamically selecting the most informative triplets during training, focusing on each mini-batch. This strategy makes the process memory-efficient by negating the need to store all possible triplet combinations. Still, it also enhances model performance through quicker convergence by focusing on challenging examples based on the current model state.

This hard triplet mining selects triplets $(a, p, n)$ to maximize the Euclidean distance between the anchor and positive samples and the anchor and negative samples. These distances, $D_{\text{AP}}$ and $D_{\text{AN}}$, are computed respectively as:


\begin{align}
D_{\text{AP}} &= \sqrt{\sum_{i} (A_i - P_i)^2} & D_{\text{AN}} &= \sqrt{\sum_{i} (A_i - N_i)^2}
\end{align}


In implementing the batch normalization step, it is necessary to standardize the audio lengths across all elements in the minibatch. We opted to zero-pad all clips to the length of the longest clip, valuing data integrity and completeness over potential performance trade-offs. Thus, the length of the longest array in the batch, which sets the standard for all others, is as follows:

\begin{equation}
L_{\text{max}} = \max_{i \in I} \left( \max \left( |A_i|, |P_i|, |N_i| \right) \right)
\end{equation}

\begin{itemize}
    \item $I$ represents the set of all items in the batch.
    \item $|A_i|$, $|P_i|$, and $|N_i|$ denote the lengths of the anchor, positive, and negative embedding vectors for the $i$-th item, respectively.
    \item The $\max$ function is applied to find the longest of these three lengths for each item, and then the maximum of these maximum lengths is taken over all items in the batch. This gives the maximum length, $L_{\text{max}}$, of any vector in the batch.
    \end{itemize}

\newpage