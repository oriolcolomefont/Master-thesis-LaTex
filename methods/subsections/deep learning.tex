\section{Deep Learning modeling}

Deep learning (DL), a subset of machine learning, employs deep neural networks (DNNs) to decipher hierarchical data representations, allowing models with multiple processing layers to abstract data across numerous levels \cite{LeCun2015DeepLearning, Sarker2021DeepDirections}. DL models have significantly improved various domains due to their nonlinear modeling capabilities and scalability with large datasets.

They leverage the central algorithm, backpropagation, which calculates gradients of the loss function related to network parameters through forward and backward pass processes, thereby optimizing model performance.

%Backpropagation
\input{figures/neural networks/Back_propagation}

These computational models somewhat rooted in biological neural systems, utilize layers of artificial neurons for tasks such as pattern recognition, classification, and regression. They evolve by adjusting connections between neurons during learning, a concept initiated in the mid-20th century but modernized with Frank Rosenblatt's single-layer perceptron in 1957, capable of learning linearly separable patterns \cite{perceptron}.

%Multi-layer_NN
\begin{figure}[ht]
	\centering
	% NEURAL NETWORK with coefficients, shifted
\begin{tikzpicture}[x=2.2cm,y=1.4cm]
  \message{^^JNeural network, shifted}
  \readlist\Nnod{4,5,5,5,3} % array of number of nodes per layer
  \readlist\Nstr{n,m,m,m,k} % array of string number of nodes per layer
  \readlist\Cstr{\strut x,a^{(\prev)},a^{(\prev)},a^{(\prev)},y} % array of coefficient symbol per layer
  \def\yshift{0.5} % shift last node for dots
  
  \message{^^J  Layer}
  \foreachitem \N \in \Nnod{ % loop over layers
    \def\lay{\Ncnt} % alias of index of current layer
    \pgfmathsetmacro\prev{int(\Ncnt-1)} % number of previous layer
    \message{\lay,}
    \foreach \i [evaluate={\c=int(\i==\N); \y=\N/2-\i-\c*\yshift;
                 \index=(\i<\N?int(\i):"\Nstr[\lay]");
                 \x=\lay; \n=\nstyle;}] in {1,...,\N}{ % loop over nodes
      % NODES
      \node[node \n] (N\lay-\i) at (\x,\y) {$\Cstr[\lay]_{\index}$};
      
      % CONNECTIONS
      \ifnum\lay>1 % connect to previous layer
        \foreach \j in {1,...,\Nnod[\prev]}{ % loop over nodes in previous layer
          \draw[connect,white,line width=1.2] (N\prev-\j) -- (N\lay-\i);
          \draw[connect] (N\prev-\j) -- (N\lay-\i);
          %\draw[connect] (N\prev-\j.0) -- (N\lay-\i.180); % connect to left
        }
      \fi % else: nothing to connect first layer
      
    }
    \path (N\lay-\N) --++ (0,1+\yshift) node[midway,scale=1.5] {$\vdots$};
  }
  
  % LABELS
  \node[above=0.8,align=center,mygreen!60!black] at (N1-1.90) {input\\[-0.2em]layer};
  \node[above=0.5,align=center,myblue!60!black] at (N3-1.90) {hidden layers};
  \node[above=1.3,align=center,myred!60!black] at (N\Nnodlen-1.90) {output\\[-0.2em]layer};
  
\end{tikzpicture}
	\caption[Network graph for perceptron.]{\small{Network graph of a perceptron with $D$ input units and $C$ output units. The $l^{\text{th}}$ hidden layer contains $m^{(l)}$ hidden units. Each neuron in a layer receives input from the previous layer and computes an output value using an activation function. The output of the last layer represents the prediction or classification result.
}}
	\label{fig: multilayer color perceptron}
\end{figure}

Another crucial component in the operation of DNNs is the activation function. This function is applied at every layer, transforming the linear combination of the input with the layer weights into an output that is passed onto the next layer. It introduces non-linearity into the network, enabling it to learn complex patterns and relationships.

%Activation equation
\begin{center}
\end{center}
\begin{figure}
    \centering
    \scalebox{1.0}{
% NEURAL NETWORK activation
\begin{tikzpicture}[x=2.7cm,y=1.6cm]
\centering
  \message{^^JNeural network activation}
  \def\NI{5} % number of nodes in input layers
  \def\NO{4} % number of nodes in output layers
  \def\yshift{0.4} % shift last node for dots
  
  % INPUT LAYER
  \foreach \i [evaluate={\c=int(\i==\NI); \y=\NI/2-\i-\c*\yshift; \index=(\i<\NI?int(\i):"n");}]
              in {1,...,\NI}{ % loop over nodes
    \node[node in,outer sep=0.6] (NI-\i) at (0,\y) {$a_{\index}^{(0)}$};
  }
  
  % OUTPUT LAYER
  \foreach \i [evaluate={\c=int(\i==\NO); \y=\NO/2-\i-\c*\yshift; \index=(\i<\NO?int(\i):"m");}]
    in {\NO,...,1}{ % loop over nodes
    \ifnum\i=1 % high-lighted node
      \node[node hidden]
        (NO-\i) at (1,\y) {$a_{\index}^{(1)}$};
      \foreach \j [evaluate={\index=(\j<\NI?int(\j):"n");}] in {1,...,\NI}{ % loop over nodes in previous layer
        \draw[connect,white,line width=1.2] (NI-\j) -- (NO-\i);
        \draw[connect] (NI-\j) -- (NO-\i)
          node[pos=0.50] {\contour{white}{$w_{1,\index}$}};
      }
    \else % other light-colored nodes
      \node[node,blue!20!black!80,draw=myblue!20,fill=myblue!5]
        (NO-\i) at (1,\y) {$a_{\index}^{(1)}$};
      \foreach \j in {1,...,\NI}{ % loop over nodes in previous layer
        %\draw[connect,white,line width=1.2] (NI-\j) -- (NO-\i);
        \draw[connect,myblue!20] (NI-\j) -- (NO-\i);
      }
    \fi
  }
  
  % DOTS
  \path (NI-\NI) --++ (0,1+\yshift) node[midway,scale=1.2] {$\vdots$};
  \path (NO-\NO) --++ (0,1+\yshift) node[midway,scale=1.2] {$\vdots$};
  
  % EQUATIONS
  \def\agr#1{{\color{mydarkgreen}a_{#1}^{(0)}}}
  \node[below=17,right=11,mydarkblue,scale=0.95] at (NO-1)
    {$\begin{aligned} %\underset{\text{bias}}{b_1}
       &= \color{mydarkred}\sigma\left( \color{black}
            w_{1,0}\agr{0} + w_{1,1}\agr{1} + \ldots + w_{1,n}\agr{n} + b_1^{(0)}
          \color{mydarkred}\right)\\
       &= \color{mydarkred}\sigma\left( \color{black}
            \sum_{i=1}^{n} w_{1,i}\agr{i} + b_1^{(0)}
           \color{mydarkred}\right)
     \end{aligned}$};
  \node[right,scale=0.9] at (1.3,-1.3)
    {$\begin{aligned}
      {\color{mydarkblue}
      \begin{pmatrix}
        a_{1}^{(1)} \\[0.3em]
        a_{2}^{(1)} \\
        \vdots \\
        a_{m}^{(1)}
      \end{pmatrix}}
      &=
      \color{mydarkred}\sigma\left[ \color{black}
      \begin{pmatrix}
        w_{1,0} & w_{1,1} & \ldots & w_{1,n} \\
        w_{2,0} & w_{2,1} & \ldots & w_{2,n} \\
        \vdots  & \vdots  & \ddots & \vdots  \\
        w_{m,0} & w_{m,1} & \ldots & w_{m,n}
      \end{pmatrix}
      {\color{mydarkgreen}
      \begin{pmatrix}
        a_{1}^{(0)} \\[0.3em]
        a_{2}^{(0)} \\
        \vdots \\
        a_{n}^{(0)}
      \end{pmatrix}}
      +
      \begin{pmatrix}
        b_{1}^{(0)} \\[0.3em]
        b_{2}^{(0)} \\
        \vdots \\
        b_{m}^{(0)}
      \end{pmatrix}
      \color{mydarkred}\right]\\[0.5em]
      {\color{mydarkblue}a^{(1)}}
      &= \color{mydarkred}\sigma\left( \color{black}
           \mathbf{W}^{(0)} {\color{mydarkgreen}a^{(0)}}+\mathbf{b}^{(0)}
         \color{mydarkred}\right)
         %\color{black},\quad \mathbf{W}^{(0)} \in \mathbb{R}^{m\times n}
    \end{aligned}$};

\end{tikzpicture}
}
\caption[Activation function \cite{tikz}.]{\small{Input $a_i^{(0)}$ and output $a_j^{(1)}$ layers are connected by weight matrix $\mathbf{W}^{(0)}$ and bias vector $\mathbf{b}^{(0)}$, processed by activation function $\sigma$.}}


    \label{fig:activation_function}

\end{figure}

\newpage