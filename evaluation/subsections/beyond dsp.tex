\section{Music Structure Analysis as downstream task}

Let's pick a core MIR task as a sample to display such complexity: Music Structure Analysis (MSA), an interdisciplinary field that aims to understand the structure of music \cite{Nieto2020Audio-BasedApplications}. However, due to subjectivity, ambiguity, and data scarcity, audio-based MSA faces challenges like boundary placement ambiguity and similarity quantification \cite{NietoPerceptualMusic}. 

The main principles of MSA were initially defined as homogeneity, novelty, and repetition, with the addition of regularity. The checkerboard kernel technique is a simple and effective method for MSA based on the homogeneity principle. The kernel, with a checkerboard-like structure, is convolved over the main diagonal of a Self-Similarity Matrix (SSM) such as $S_{ij}$ is the similarity between time frames $i$ and $j$, $\vec{x}_i$ is the vector representation of time frame $i$, and $\left| \vec{x}_i \right|$ is the Euclidean norm of vector $\vec{x}_i$. The dot product of two vectors $\vec{x}_i$ and $\vec{x}_j$ is denoted by $\vec{x}_i \cdot \vec{x}_j$.

\begin{equation}
\label{eq:segment_similarity}
S_{ij} = \frac{\vec{x}_i \cdot \vec{x}_j}{\left\| \vec{x}_i \right\| \left\| \vec{x}_j \right\|}
\end{equation}

This yields a novelty curve highlighting sudden changes in the selected musical features from which to extract the segment boundaries.

\begin{equation}
n_i = \frac{\sum\limits_{j=1}^{N} S_{ij} - N\cdot S_{ii}}{\sqrt{\sum\limits_{j=1}^{N} (S_{ij} - S_{ii})^2}}
\end{equation}

Where $N$ is the number of time frames, $S_{ij}$ is the similarity between time frames $i$ and $j$, $S_{ii}$ is the similarity between time frame $i$ and itself, and $n_i$ is the novelty value for time frame $i$.

The numerator of the expression represents the sum of similarities between time frame $i$ and all other time frames minus the average similarity across all time frames. The denominator of the expression is the standard deviation of the similarities, which is used to normalize the values to ensure that the novelty values are between -1 and 1.


Mathematical operations are used on predefined principles such as homogeneity and repetition to analyze a musical structure. It explores features extracted from a musical signal and identifies peaks in a novelty curve to extract segment boundaries. In contrast, human aural skills rely on the perception and interpretation of music by human listeners based on cultural and academic background. They can capture subtle nuances of musical structure, but those still are subjective and can vary across listeners.

Combining technical tasks, aural skills, and music perception is challenging due to the complexity and subjectivity of musical perception, discrepancies between the features extracted by DSP techniques and the features perceived by humans, and the variability of musical signals and individual differences in musical perception. 

Purely mathematical approaches may not always reflect how humans perceive music and may not capture the subtle variations in timing or harmonic relationships critical to musical perception. Variability in musical signals due to instrumentation, genre, and performance style further complicates the development of these techniques that can be generalized to different musical contexts.


% A Bloch sphere of radius |a| = 1 contains all possible states of a two-state quantum system (qubit).
% Each Bloch vector fully determines a spin-1/2 density matrix.
% Used in Exercise Sheet 10 of Statistical Physics by Manfred Salmhofer (2016), available at https://janosh.dev/physics/statistical-physics.


%\usetikzlibrary{angles, quotes}

\begin{figure}
    \centering
    \scalebox{1.0}{
    \begin{tikzpicture}

  % Define radius
  \def\r{3}

  % Bloch vector
  \draw (0, 0) node[circle, fill, inner sep=1] (orig) {} -- (\r/3, \r/2) node[circle, fill, inner sep=0.7, label=above:$\vec{a}$] (a) {};
  \draw[dashed] (orig) -- (\r/3, -\r/5) node (phi) {} -- (a);

  % Sphere
  \draw (orig) circle (\r);
  \draw[dashed] (orig) ellipse (\r{} and \r/3);

  % Axes
  \draw[->] (orig) -- ++(-\r/5, -\r/3) node[below] (x1) {$x_1$};
  \draw[->] (orig) -- ++(\r, 0) node[right] (x2) {$x_n \in \mathbb{R}$ $2 \leq n \leq 127$};
  \draw[->] (orig) -- ++(0, \r) node[above] (x3) {$x_128$};

\end{tikzpicture}}
    \caption[128-dimensional hypersphere]{\small{A representation of a 128-dimensional hypersphere in the Euclidean space $\mathbb{R}^{128}$. Due to the limitations of visualizing high-dimensional objects in two or three dimensions, this figure offers a simplified and abstract depiction of the hypersphere.}}
    \label{fig:my_label}
\end{figure}