\section{Related work}

Neural networks have recently facilitated the development of low-dimensional embeddings that capture essential musical features, enabling the computation of various task-specific aspects. These aspects include:

\begin{itemize}

\item The utilization of multi-level music segmentation through deep audio embeddings to replace manually engineered features \cite{SalamonDeepSegmentation}.

\item The enhancement of environmental sound classification \cite{Kim2020OneStrategies}.

\item The improvement of vocal-centric music tasks using cross-domain audio embedding \cite{Kim2021LearningLoss}.

\item The enhancement of audio classification through a combination of task-specific and pre-trained features \cite{Hung2022Feature-informedClassification}.

\item The creation of a music similarity search engine specifically for video producers \cite{epidemic}.

\item The improved performance in Music Emotion Recognition (MER) tasks, eliminating the necessity for expert human engineering \cite{KohComparisonRecognition}.

\item The resolution of the cross-modal text-to-music retrieval problem, enabling content creators to find music that matches the emotion conveyed in their text-based stories \cite{WonEmotionStories}.

\item Music rearrangement is traditionally undertaken by expert music engineers or more classical MIR approaches \cite{Stoller2018IntuitiveTransitions}. This innovative approach presents a promising avenue for automated music rearrangement, enhancing the process's efficiency and quality. \cite{Plachouras2023MusicSegmentation}

\end{itemize}

Moreover, deep audio embeddings offer the advantage of transferability. Once trained, these latent representations can be utilized as a starting point for various other tasks, thus saving computational resources and time compared to training a model from scratch \cite{HamelTransferSimilarity}.

On top of that and returning to musicological influences, the middle ground in Schenkerian analysis can be compared to the mesostructure in music. Analogous to the mesostructure, it represents the intermediate-level musical patterns that bridge the micro and macro structures. This element is often overlooked when applying deep learning to music. As emphasized by \cite{Mesostructures2023}, models that address this mesostructure level could significantly enrich tasks such as music analysis, composition, and retrieval, providing a more holistic understanding of music similar to the insights gained from analyzing the middle-ground in Schenkerian analysis. \cite{Introduction_to_Schenkerian_Analysis}

Given their demonstrated success in the literature and their potential for transfer learning, exploring deep audio embeddings for MIR downstream tasks seems worthwhile.
