\section{Objectives}

This study aims to explore the core tonal structure of Western music by constructing and analyzing an embedding space applicable to subsequent Music Information Retrieval (MIR) tasks. 

By employing a blend of MIR computational techniques and a musicological approach, it intends to examine a wide spectrum of musical works across diverse styles and genres. Crucial to this is the understanding of aural skills – perceptual and cognitive abilities essential for interpreting musical works through listening. 

The research focuses on the learning of a single embedding, $e(M)$, for a music piece $M$ which encapsulates the underlying high-level content such as melody contour, harmony inter-relationships, and other holistic elements. The structuring of the embedding space will prioritize the clustering of similar content and sonic quality features while maintaining a clear distinction between them. 

The goal is to create an embedding $e(M)$ that captures high-level musical content and remains agnostic to production features.