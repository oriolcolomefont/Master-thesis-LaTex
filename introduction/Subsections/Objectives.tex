\section{Objectives}

With the advent of Machine Learning (ML), MIR has expanded to encompass musically motivated neural network architectures. These networks are designed to incorporate musical concepts and perceptual approaches and use this knowledge to analyze and understand the structure of musical pieces in a way that goes beyond traditional music analysis techniques.

By studying music structure analysis using musically motivated neural networks, we will gain an in-depth understanding of the latest advancements in the field and how these techniques are being used to revolutionize our understanding of music. Additionally, we will be able to contribute to developing new and innovative methods for musical analysis.

This work aims to generalize the fundamental tonal-working structure of western-tradition music by learning and embedding space used to facilitate its application in downstream MIR tasks. Understanding how music is organized and composed is critical, and music theory plays a vital role.

In recent years, using neural networks has enabled low-dimensional musical embeddings that capture crucial musical features and allow the computation of content-based similarity between musical passages. By acquiring an embedding function $f(x) \in \mathbb{R}^d$ that maps a given course $x$ to a vector in the real space $\mathbb{R}^d$, we can measure the similarity between two musical passages using the Euclidean distance, as represented by the formula $D_{i,j} = ||f(x_i) - f(x_j)||^2$. The effectiveness of such systems depends on the embedding space's ability to interpret music in a way that aligns with human perception. Therefore, a suitable embedding space is crucial to ensure the output aligns with human expectations.

This study aims to comprehensively analyze musical works from diverse styles and genres by incorporating computational techniques from Music Information Retrieval (MIR) and a musicological perspective to contextualize the findings. Aural skills are required to comprehend musical works through listening, as they involve perceptual and cognitive abilities. The underlying musical structure refers to the systematic arrangement of elements that produce a coherent and expressive musical work. This study aims to offer fresh insights into these components of musical composition and advance our understanding of their relationships.

Let's denote the single embedding for a music piece as $e(M)$.

Given a music piece $M$, we aim to learn a single embedding, $e(M)$, simultaneously capturing the underlying content (melody, harmony, and critical, holistic elements) and the production aspects. The embedding space should be structured so that similar content and production features are clustered together, effectively isolating the two aspects. Formally, our objective is to learn the embedding function $e(M)$ such that:

\begin{itemize}
  \item Similar content features are closer in the embedding space: $\|e(M_1) - e(M_2)\| < \epsilon_c$ for music pieces $M_1$ and $M_2$ with similar content.
  \item Similar production features are closer in the embedding space: $\|e(M_1) - e(M_3)\| < \epsilon_p$ for music pieces $M_1$ and $M_3$ with similar production.
  \item Distinct content and production features are further apart in the embedding space.
\end{itemize}

We describe learning a single embedding $e(M)$ that captures content and production features. The embedding space is structured to cluster similar content and production features together, effectively isolating the two aspects.

Music is a complex and multi-faceted phenomenon, and implementing a model mimicking human aural music skills is challenging. However, the latest deep learning techniques provide a promising avenue to build a model that achieves good results for a specific task in music. Therefore, this study will utilize deep learning techniques to investigate the correlation between aural skills and the underlying tonal-working structure of music.

Furthermore, this study will argue that MIR research should shift from a purely technical-centered approach to a more balanced and comprehensive one incorporating other relevant domains such as musicology and music theory. By doing so, we can better understand the interplay between technical and musical aspects of music and improve our understanding of music signals and the music itself.

%%%%%%%%%%%%%%%%%%%%%%%%
%POTENTIAL APPLICATIONS
%%%%%%%%%%%%%%%%%%%%%%%%%

A self-supervised model that learns an embedding space to retrieve musical content from raw audio signals could have several potential applications and downstream tasks in Music Information Retrieval (MIR). Here are some examples:

\begin{enumerate}
\item Music recommendation: The learned embedding space can retrieve similar musical content and recommend songs or playlists to users based on their listening history or preferences.
\item Music classification: The embedding space can classify songs based on genre, mood, or other musical characteristics.
\item Music transcription: The learned embedding space can transcribe audio signals into symbolic music notation, such as MIDI or sheet music.
\item Music generation: The embedding space can generate new musical content by sampling from the learned distribution of musical features.
\item Music synchronization: The learned embedding space can synchronize audio and visual content in applications such as music videos or virtual reality experiences.
\item Music search: The embedding space can search for specific musical content based on high-level perceptual concepts and the relationships they have among them
\end{enumerate}

