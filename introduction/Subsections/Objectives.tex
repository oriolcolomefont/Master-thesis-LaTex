\section{Objectives}

This work aims to generalize the fundamental tonal-working structure of western-tradition music by learning and embedding space used to facilitate its application in downstream MIR tasks. Understanding how music is organized and composed is critical, and music theory plays a vital role in this regard.

In recent years, the utilization of neural networks has made it possible to create low-dimensional musical embeddings that capture crucial musical features and enable the computation of content-based similarity between musical passages. By acquiring an embedding function $f(x) \in \mathbb{R}^d$ that maps a given passage $x$ to a vector in the real space $\mathbb{R}^d$, researchers can measure the semantic similarity between two musical passages using the Euclidean distance, as represented by the formula $D_{i,j} = ||f(x_i) - f(x_j)||^2$. The effectiveness of such systems depends on the embedding space's ability to interpret music in a way that aligns with human perception. Therefore, a suitable embedding space is crucial to ensure that the output aligns with human expectations.

The thesis under discussion explores the various musical elements that constitute music structure analysis, which includes melody, harmony, rhythm, and form. Aural skills are required to comprehend musical works through listening, as they involve perceptual and cognitive abilities. The underlying musical structure refers to the systematic arrangement of musical elements that results in a coherent and expressive musical work.

To provide a comprehensive analysis of musical works from diverse styles and genres, the thesis employs computational techniques from MIR and incorporates a musicological perspective to contextualize the findings. The objective is to offer fresh insights into these components of musical composition and advance our understanding of their relationships.


Music is a complex and multi-faceted phenomenon, and implementing a model that mimics human aural music skills is challenging. However, using the latest deep learning techniques, we aim to build a model that achieves good results for a specific task in the domain of music.

We will argue that MIR research should shift from a purely technical-centered approach to a more balanced and comprehensive one that incorporates other relevant domains.

The study investigates the correlation between XXX, aural skills, and the underlying musical structure in musical composition. Through this research, the thesis aims to provide a better understanding of musical composition components, their relationships, and their practical applications.