\section{Motivation}

This thesis, a collaboration between \href{https://www.epidemicsound.com/}{Epidemic Sound} (ES) and the \href{https://www.upf.edu/web/mtg}{Music Technology Group} (MTG) at Universitat Pompeu Fabra (UPF), has been driven by personal, industrial, and academic motivations. 

ES, a Swedish company, curates a vast global library of over 40,000 royalty-free music tracks and 90,000+ sound effects. Based in Barcelona, the MTG specializes in innovative sound and music technologies, such as information retrieval, digital signal processing, interactive music systems, and computational musicology.

The collaboration aims to deepen our understanding of music's fundamental structures, which could enhance ES's technical offerings and further advancements in MIR research.

As a musician, my passion for music has driven me to investigate the foundational elements of musical composition, various creation techniques, and their intricate relationships. I am dedicated to extracting valuable information embedded within music across all domains, focusing on audio and sheet music (symbolic domain). Musicians who endeavor to bridge the gap and uncover musical truths within the tonal paradigm, such as Heinrich Schenker \cite{Komar1959SchenkersStructure}, or those who challenge it, like Arnold Schoenberg \cite{Samson1974SchoenbergsMusic}, George Russell \cite{LydianRussell}, and Ernst Levy \cite{LevyAHarmony}, have been a continual source of inspiration. They aim to identify abstract concepts that reinforce or disrupt the tonal foundation, advancing the tonal landscape and providing a solid base for musicians' growth, development, and understanding of tonal and atonal paradigms.

Advancements in AI research, building on the theoretical groundwork laid by pioneers such as Alan Turing and Claude Shannon, have led to significant breakthroughs \cite{Vaswani2017AttentionNeed}. With AI being applied broadly in skyrocketing popular tech products \cite{OpenAI2023GPT-4Report}, researchers and corporations must stay abreast of this rapidly evolving field. 

The emergence of self-supervised models capable of learning embedding spaces to retrieve musical content from audio signals presents new opportunities. These models, which autonomously extract information from audio data, can potentially transform multiple aspects of the music industry. Industrially, these embedding spaces can be used to devise innovative products, enhance user experiences for content creators, and stimulate innovation and collaboration across the industry.

