\section{Motivation}

we argue that MIR research needs to incorporate a more balanced approach that considers the interdisciplinary nature of music and the importance of other domains beyond DSP

As a music lover, professional musician, and connoisseur, I am interested in gaining a deeper understanding of the fundamental building of musical composition, the various techniques used to create musical works, and capturing underlying relationships among them. I am deeply interested in learning how to retrieve embedded information within music. The study of music structure analysis provides a means to systematically analyze and understand musical pieces. It has traditionally been approached through music theory, formal musical notation, musical analysis techniques, and machine learning, which I plan to blend.

With the advent of Artificial Intelligence (AI) and Machine Learning (ML), Music Information Retrieval (MIR) has expanded to encompass musically motivated neural network architectures. These networks are designed to incorporate musical concepts and perceptual approaches and use this knowledge to analyze and understand the structure of musical pieces in a way that goes beyond traditional music analysis techniques.

As a professional musician and music transcriber, I am interested in understanding how Deep Neural Networks (DNNs) filter musical features. In other words, I am eager to learn how DNNs listen to music.

By studying music structure analysis using musically motivated neural networks, we will gain an in-depth understanding of the latest advancements in the field and how these techniques are being used to revolutionize our understanding of music. Additionally, we will be able to contribute to developing new and innovative methods for musical analysis.