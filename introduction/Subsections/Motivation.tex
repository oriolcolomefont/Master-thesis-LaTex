\section{Motivation}

We argue that MIR research needs to incorporate a more balanced approach that considers the interdisciplinary nature of music and the importance of other domains beyond DSP.

Let's pick a core MSA task as a sample to display such complexity: Music Structure Analysis (MSA) is an interdisciplinary field that aims to understand the structure of music. However, due to subjectivity, ambiguity, and data scarcity, audio-based MSA faces challenges like boundary placement ambiguity and similarity quantification. The main principles of MSA were initially defined as homogeneity, novelty, and repetition, with the addition of regularity.\cite{CLMR2021}

The checkerboard kernel technique is a simple and effective method for Music Structure Analysis (MSA) based on the homogeneity principle. The kernel, with a checkerboard-like structure, is convolved over the main diagonal of a Self-Similarity Matrix (SSM) such as $S_{ij}$ is the similarity between time frames $i$ and $j$, $\vec{x}_i$ is the vector representation of time frame $i$, and $\left| \vec{x}_i \right|$ is the Euclidean norm of vector $\vec{x}_i$. The dot product of two vectors $\vec{x}_i$ and $\vec{x}_j$ is denoted by $\vec{x}_i \cdot \vec{x}_j$.

\begin{equation}
\label{eq:segment_similarity}
S_{ij} = \frac{\vec{x}_i \cdot \vec{x}_j}{\left\| \vec{x}_i \right\| \left\| \vec{x}_j \right\|}
\end{equation}

This yields a novelty curve that highlights sudden changes in the selected musical features (some examples would be chroma, MFCCs, beat-synchronous features, onset and offset detection, and harmonic and percussive separation) from which to extract the segment boundaries.

\begin{equation}
n_i = \frac{\sum\limits_{j=1}^{N} S_{ij} - N\cdot S_{ii}}{\sqrt{\sum\limits_{j=1}^{N} (S_{ij} - S_{ii})^2}}
\end{equation}

Where $N$ is the number of time frames, $S_{ij}$ is the similarity between time frames $i$ and $j$, $S_{ii}$ is the similarity between time frame $i$ and itself, and $n_i$ is the novelty value for time frame $i$.

The numerator of the expression represents the sum of similarities between time frame $i$ and all other time frames minus the average similarity across all time frames. The denominator of the expression is the standard deviation of the similarities, which is used to normalize the values to ensure that the novelty values are between -1 and 1.


Mathematical operations are used on predefined principles such as homogeneity and repetition to analyze a musical structure. It analyzes features extracted from a musical signal and identifies peaks in a novelty curve to extract segment boundaries. In contrast, human aural skills rely on the perception and interpretation of music by human listeners based on cultural and academic background. They can capture subtle nuances of musical structure, but those still are subjective and can vary across listeners.

Combining technical tasks, aural skills, and music perception is challenging due to the complexity and subjectivity of musical perception, discrepancies between the features extracted by DSP techniques and the features perceived by humans, and the variability of musical signals and individual differences in musical perception. DSP techniques may not always reflect how humans perceive music and may not capture the subtle variations in timing or harmonic relationships critical to musical perception. Variability in musical signals due to instrumentation, genre, and performance style further complicates the development of DSP techniques that can be generalized to different musical contexts.

As a music lover, professional musician, and connoisseur, I am interested in gaining a deeper understanding of the fundamental building of musical composition, the various techniques used to create musical works, and capturing underlying relationships among them using DN.  I am eager to learn how DNNs listen to music. I am deeply interested in learning how to retrieve embedded information within music. 