\chapter{Introduction}

Music is an essential part of human culture, history, and society. It has existed since ancient times and evolved, reflecting changes in our cultures, traditions, and beliefs. Despite its importance, the study of music has always been a challenging and complex task due to its subjective nature and the vast array of musical styles, genres, and cultures that exist worldwide.

Humanity has been studying music to understand its underlying structure for millennia. From the ancient Greeks to the early days, music has been a subject of fascination and inquiry. This rich music theory and analysis history provide a valuable foundation for guiding and evaluating Music Information Retrieval (MIR) algorithms and models. 

We all humbly seek to unlock the secrets of music's complex and multifaceted nature.

MIR is a research area that has emerged in the last decades as an interdisciplinary field combining engineering, musicology, and neuroscience. MIR aims to develop algorithms and techniques to automatically analyze, classify, and retrieve musical information from mainly audio signals. I believe that traditionally, MIR research has primarily concentrated on technical aspects, such as Digital Signal Processing (DSP) or pattern recognition. As a result, other relevant domains, including perception, cognition, music theory, aural skills, and musicology, along with its subfields, might have been overlooked. 

It is worth mentioning that there has been a significant shift in this tendency on retrieval tasks in the symbolic and musicological domain, such as powerful open-source analysis frameworks \cite{ScottCuthbertChristopherArizaMusic21:Data}, MIDI to proper sheet music syntax \cite{Suzuki2021ScoreRepresentation}, projective orchestration \cite{Crestel2018AOrchestration}, or automatic lead sheet generation \cite{WeilAutomaticSignals} to name a few.

The reason behind this is non-other but the implementation complexity for such heuristics and abstract fields of study. Closing the gap between human perception, environmental and cultural factors, and computational solutions to mimic the formers is an already tackled topic \cite{Large1994ResonanceMeter}\cite{Mullensiefen2003MeasuringJudgments}\cite{Sears2014PerceivingCadence}\cite{Thaut2014HumanPatter}\cite{, yet remains a vast open challenge.