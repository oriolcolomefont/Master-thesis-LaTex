\chapter{Introduction}

Music is a complex and fascinating art form studied by scholars and musicians for centuries. Music theory plays a crucial role in understanding the underlying principles of how music is organized and composed, and analyzing the structure of music is a challenging task. 

Recently, Music Structure Analysis (MSA) has emerged as a subfield of Music Information Retrieval (MIR), aiming to develop computational methods to analyze the structure of music. This master thesis investigates the relationship between MSA, aural skills, and the underlying musical structure in musical composition. 

In recent years, the utilization of neural networks has allowed for the creation of low-dimensional musical embeddings that capture important musical features and enable the computation of semantic similarity between musical passages. By learning an embedding function $f(x) \in \mathbb{R}^d$ that maps a given passage $x$ to a vector in the real space $\mathbb{R}^d$, researchers can quantify the semantic similarity between two musical passages using the Euclidean distance, which is represented by the formula $D_{i,j} = ||f(x_i) - f(x_j)||^2$. Adequate embedding space is essential in ensuring that the generated output aligns with human expectations, especially in interactive applications where a human performer influences the machine's response. The success of such systems depends on the embedding space's ability to interpret music in a way that aligns with human perception.

The thesis explores the musical elements that constitute music\footnote{The term "music" used in this context refers specifically to Western tonal music tradition. It is assumed that the musical structures and elements discussed are based on this tradition and may not apply to other musical styles or cultures.} structure analysis, including melody, harmony, rhythm, and form. Aural skills involve perceptual and cognitive abilities that enable individuals to comprehend musical works through listening. The underlying musical structure refers to the systematic arrangement of musical elements that results in a coherent and expressive musical work. 

This thesis provides a comprehensive analysis of musical works from diverse styles and genres, utilizing computational techniques from MIR and incorporating a musicological perspective to contextualize the findings. 

This research aims to offer fresh insights into these components of musical composition and advance our understanding of their relationships.