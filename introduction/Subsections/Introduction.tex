\chapter{Introduction}

Music is essential to human culture, history, and society; it has existed since ancient times. It reflects the changes in our cultures, traditions, and beliefs as they evolve \footnote{In this thesis, "music" refers to the Western tradition, characterized by conventions, practices, and aesthetics that have developed primarily in Europe and North America. This usage does not deny the diversity and richness of other musical traditions worldwide but reflects the present study's focus on a specific cultural context.}. Studying music, therefore, has always been a challenging and complex task due to its subjective nature, relative and mutable ground truth, and the sheer variety of musical styles, genres, and worldwide cultures.

For millennia, people have sought to understand the underlying structure of music. Music has captivated and intrigued us from the ancient Greeks to the present day. This rich music theory and analysis history form a valuable foundation for guiding and evaluating MIR algorithms and models.

MIR has emerged over the last few decades as an interdisciplinary field that combines engineering, musicology, and neuroscience. It aims to develop algorithms and techniques for automatically analyzing, classifying, and retrieving musical information, primarily from audio signals. 