\chapter{Introduction}

Music is an essential part of human culture, history, and society. It has existed in our lives since ancient times and has evolved, reflecting changes in our cultures, traditions, and beliefs. Despite its importance, the study of music has always been a challenging and complex task due to its subjective nature and the vast array of musical styles, genres, and cultures that exist worldwide.

Humanity has been studying music to understand its underlying structure for millennia. From the ancient Greeks, who believed that music had a mathematical and cosmic harmony, to the early works of theorists such as Rameau, Fux, and Schenker, and more recent approaches such as generative theory and corpus studies, music has been a subject of fascination and inquiry. This rich music theory and analysis history provides a valuable foundation for guiding and evaluating MIR algorithms and models. 

We all humbly seek to unlock the secrets of music's complex and multifaceted nature.

Music Information Retrieval (MIR) is a research area that has emerged in the last decades as an interdisciplinary field combining engineering, musicology, and neuroscience. MIR aims to develop algorithms and techniques to automatically analyze, classify, and retrieve musical information from audio signals. In my opinion, MIR research has been traditionally focused on technical aspects, such as digital signal processing (DSP), machine learning, and pattern recognition, neglecting other relevant domains, such as perception, cognition, music theory, aural skills, and musicology.

The reason behind this is non-other but the high implementation complexity for such heuristics and abstract fields of study.

Let's pick a core MSA task as a sample to display such complexity: Music Structure Analysis (MSA) is an interdisciplinary field that aims to understand the structure of music. However, due to subjectivity, ambiguity, and data scarcity, audio-based MSA faces challenges such as boundary placement ambiguity and similarity quantification. The main principles of MSA were initially defined as homogeneity, novelty, and repetition, with the addition of regularity.

The checkerboard kernel technique is a simple and effective method for Music Structure Analysis (MSA) based on the homogeneity principle. The kernel, with a checkerboard-like structure, is convolved over the main diagonal of a Self-Similarity Matrix (SSM) such as $S_{ij}$ is the similarity between time frames $i$ and $j$, $\vec{x}_i$ is the vector representation of time frame $i$, and $\left| \vec{x}_i \right|$ is the Euclidean norm of vector $\vec{x}_i$. The dot product of two vectors $\vec{x}_i$ and $\vec{x}_j$ is denoted by $\vec{x}_i \cdot \vec{x}_j$.

\begin{equation}
\label{eq:segment_similarity}
S_{ij} = \frac{\vec{x}_i \cdot \vec{x}_j}{\left\| \vec{x}_i \right\| \left\| \vec{x}_j \right\|}
\end{equation}

This yields a novelty curve that highlights sudden changes in the selected musical features (e.g. chroma, MFCCs, beat-synchronous features, onset and offset detection, and harmonic and percussive separation) from which to extract the segment boundaries.

\begin{equation}
n_i = \frac{\sum\limits_{j=1}^{N} S_{ij} - N\cdot S_{ii}}{\sqrt{\sum\limits_{j=1}^{N} (S_{ij} - S_{ii})^2}}
\end{equation}

Where $N$ is the number of time frames, $S_{ij}$ is the similarity between time frames $i$ and $j$, $S_{ii}$ is the similarity between time frame $i$ and itself, and $n_i$ is the novelty value for time frame $i$.

The numerator of the expression represents the sum of similarities between time frame $i$ and all other time frames minus the average similarity across all time frames. The denominator of the expression is the standard deviation of the similarities, which is used to normalize the values to ensure that the novelty values are between -1 and 1.


Mathematical operations are used on predefined principles such as homogeneity and repetition to analyze a musical structure. It relies on analyzing features extracted from a musical signal and identifying peaks in a novelty curve to extract segment boundaries. In contrast, human aural skills rely on the perception and interpretation of music by human listeners based on cultural and academical background and can capture subtle nuances of musical structure, but those still are subjective and can vary across listeners.

Combining technical tasks, aural skills, and music perception is challenging due to the complexity and subjectivity of musical perception, discrepancies between the features extracted by DSP techniques and the features perceived by humans, and the variability of musical signals and individual differences in musical perception. DSP techniques may not always reflect the way humans perceive music, and may not capture the subtle variations in timing or harmonic relationships that are critical to musical perception. Variability in musical signals due to instrumentation, genre, and performance style further complicates the development of DSP techniques that can be generalized to different musical contexts.

