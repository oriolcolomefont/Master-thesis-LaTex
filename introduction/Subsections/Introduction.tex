\chapter{Introduction}

Music, an essential component of human culture, history, and society, has existed since ancient times. It reflects the changes in our cultures, traditions, and beliefs as it evolves. Studying music, however, has always been a challenging and complex task due to its subjective nature and the sheer variety of musical styles, genres, and cultures that exist worldwide.

For millennia, people have sought to understand the underlying structure of music. Music has captivated and intrigued us from the ancient Greeks to the present day. This rich music theory and analysis history form a valuable foundation for guiding and evaluating Music Information Retrieval (MIR) algorithms and models.

Music Information Retrieval (MIR) has emerged over the last few decades as an interdisciplinary field that combines engineering, musicology, and neuroscience. It aims to develop algorithms and techniques for automatically analyzing, classifying, and retrieving musical information, primarily from audio signals. While MIR research has traditionally focused on technical aspects such as Digital Signal Processing (DSP) and pattern recognition, it may have overlooked other relevant domains, including perception, cognition, music theory, aural skills, and musicology, along with their subfields.

\subsubsection{Bridging the gap}

Music similarity is a critical factor in query-by-example retrieval systems. However, similarity calculation and definition can differ depending on user needs and retrieval scenarios. While acoustic properties like voice, genre, melody, tempo, instrumentation, and rhythm contribute to similarity, personal preferences, and cultural context also play a significant role. For instance, a user may avoid a particular artist's music despite its similarity to their favorite artists due to the album cover. Moreover, the user's context, such as driving or running, can also influence their preference and perception of similarity. 

Metadata is the traditional approach to electronic music searches, but it can be impractical and unintuitive for users who are unfamiliar with most tracks. Retrieval systems often overlook the cultural, personal, and musical aspects of music information needs, rendering them less useful for non-expert users. Many people search for music based on emotions or usage context rather than bibliographic information.