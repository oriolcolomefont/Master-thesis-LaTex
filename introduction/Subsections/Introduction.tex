\chapter{Introduction}

Music, an essential component of human culture, history, and society, has existed since ancient times. It reflects the changes in our cultures, traditions, and beliefs as it evolves. Studying music, however, has always been a challenging and complex task due to its subjective nature and the sheer variety of musical styles, genres, and cultures that exist worldwide.

For millennia, people have sought to understand the underlying structure of music. Music has captivated and intrigued us from the ancient Greeks to the present day. This rich music theory and analysis history form a valuable foundation for guiding and evaluating Music Information Retrieval (MIR) algorithms and models.

Music Information Retrieval (MIR) has emerged over the last few decades as an interdisciplinary field that combines engineering, musicology, and neuroscience. It aims to develop algorithms and techniques for automatically analyzing, classifying, and retrieving musical information, primarily from audio signals. While MIR research has traditionally focused on technical aspects such as Digital Signal Processing (DSP) and pattern recognition, it may have overlooked other relevant domains, including perception, cognition, music theory, aural skills, and musicology, along with their subfields.

\subsubsection{Bridging the gap}

There has been a noticeable shift in this tendency in recent years, with a growing emphasis on retrieval tasks in the symbolic and musicological domain. Several powerful open-source analysis frameworks have emerged for the symbolic domain \cite{ScottCuthbertChristopherArizaMusic21:Data}, as well as advancements in MIDI to human-readable sheet music syntax \cite{Suzuki2021ScoreRepresentation}, projective orchestration \cite{Crestel2018AOrchestration}, and automatic lead sheet generation \cite{WeilAutomaticSignals}, among others.

The driving force behind this shift might be the recognition of the complexity of implementing heuristics and abstract fields of study. Bridging the gap between human perception \cite{Large1994ResonanceMeter}\cite{Thaut2014HumanPatter}, environmental and cultural factors \cite{Sears2014PerceivingCadence}, and computational solutions that mimic these aspects \cite{Mullensiefen2003MeasuringJudgments} is a topic that has been previously explored \cite{Hernandez-Olivan2023SymbolicMethods}. 

Still, it remains a vast and open challenge.