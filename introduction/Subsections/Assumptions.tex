\section{Assumptions}

This thesis assumes that high-level musical concepts are agnostic and invariant to the unique tonal quality of the waveform, much like how a sheet music composition can be performed in infinite ways using different instruments and styles while retaining its musical essence. 

We rely on a flexible and adaptable blueprint for musical expression, allowing performers to interpret and express the same musical concepts differently creatively. 

Different waveforms can express the same musical concepts despite having distinct timbres or production styles. Still, it is crucial to distinguish between low-level and high-level musical features when analyzing or comparing different musical pieces. High-level features, such as melody, harmony, rhythm, tempo, form, and expression, are more abstract and can be recognized regardless of the specific waveform or production style. 

The thesis relies on the objective and comprehensive perspective on a piece's musical content and meaning, much like reading sheet music can reveal the underlying musical structure and intention behind a piece of music. 

One sheet music composition can be performed and recorded in a limitless variety of ways using different instruments, voices, tempos, and interpretations, resulting in an infinite number of valid waveforms representing its musical content.

Conversely, the same waveform may have only a limited number of sheet music representations that can accurately capture its musical elements as per the current tradition and conventions. This is because the unique tonal quality of the waveform is strongly influenced by the specific instrument or production technology used to create it, making it difficult to transcribe its sonic properties into a precise sheet music notation. Moreover, this unique tonal quality is represented by high-level natural language concepts belonging to a rich old cultural and sociological tradition.

