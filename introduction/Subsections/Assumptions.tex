\section{Assumptions}

This thesis argues that high-level musical concepts are invariant to the unique tonal quality of the waveform, much like a sheet music composition can be performed in countless ways using various instruments and styles while retaining its essence. 

High-level features like melody, harmony, rhythm, tempo, form, and expression are more abstract and can be recognized regardless of waveform or production style. This approach provides an objective and comprehensive perspective on a piece's musical content and meaning, similar to how reading sheet music reveals a piece's underlying structure and intention. 

While a waveform may have a limited number of sheet music representations that accurately capture its musical elements, one sheet music composition can be performed in infinite ways using different instruments, voices, tempos, and interpretations. The unique tonal quality of a waveform is strongly influenced by the specific instrument or production technology used to create it, making it difficult to transcribe its sonic properties into precise sheet music notation. 

Additionally, this unique tonal quality is represented by high-level natural language concepts that belong to a rich cultural and sociological tradition.

