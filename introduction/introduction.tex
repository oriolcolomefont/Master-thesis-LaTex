\normallinespacing

\chapter{Introduction}

Music\footnote{The term "music" used in this context refers specifically to Western tonal music tradition. It is assumed that the musical structures and elements discussed are based on this tradition and may not apply to other musical styles or cultures.} Structure Analysis (MSA) is the field of study that aims to understand music's underlying organization and structure.

MSA is a multidisciplinary field that draws on techniques from music theory, signal processing, machine learning, and other areas. It can be applied to various musical styles and genres and has been used for tasks such as music transcription, generation, and similarity. 

\section{Motivation}

As a music lover and connoisseur, I am interested in gaining a deeper understanding of the fundamental building of musical composition, the various techniques used to create musical works, and the underlying relationship among them. I am deeply interested in learning how to retrieve embedded information within music. The study of music structure analysis provides a means to systematically analyze and understand musical pieces. It has traditionally been approached through music theory, formal musical notation, musical analysis techniques, and machine learning, which I plan to blend.

With the advent of artificial intelligence and machine learning, Music Information Retrieval (MIR) has expanded to encompass musically motivated neural network architectures. These networks are designed to incorporate musical concepts and perceptual approaches and use this knowledge to analyze and understand the structure of musical pieces in a way that goes beyond traditional music analysis techniques.

As a professional music transcriber, I am interested in understanding how deep neural networks (DNNs) filter musical features. In other words, I am eager to learn how DNNs listen~\cite{7500246} to music.

By studying music structure analysis using musically motivated neural networks, you will gain an in-depth understanding of the latest advancements in the field and how these techniques are being used to revolutionize our understanding of music. Additionally, we will be able to contribute to developing new and innovative methods for musical analysis~\cite{Huang2019MusicTG}

\section{Objectives}

As a professional music transcriber, I aim to understand how deep neural networks (DNNs) filter musical features, listen to the NN layers from a musician's perspective and analyze the resulting data. I aim to approach the technology from a technical and musical standpoint to understand how DNNs process and extract musical information. By doing so, I aim to enhance the accuracy and efficiency of the process and contribute to developing AI technology for music analysis. By combining my expertise in music with my -not fully developed- knowledge of DNNs, I hope to bring a unique perspective to the field and contribute to the advancement of music analysis.

\section{Structure of the Report}

In this report, we have organized the information into several sections and sub-sections to provide a clear and comprehensive overview of the topic. The structure of the report is as follows:

\begin{itemize}
\item Introduction
  \begin{itemize}
  \item Purpose
  \item Methodology
  \item Scope
  \end{itemize}
\item Main Body
  \begin{itemize}
  \item Section 1
  \item Section 2
  \item Section 3
  \end{itemize}
\item Conclusion
  \begin{itemize}
  \item Summary
  \item Recommendations
  \end{itemize}
\item References
  \begin{itemize}
  \item Sources Cited
  \end{itemize}
\end{itemize}

\newpage


