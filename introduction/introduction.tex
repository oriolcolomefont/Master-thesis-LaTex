\normallinespacing

\chapter{Introduction}

Music is a complex and fascinating art form studied by scholars and musicians for centuries. Music theory plays a crucial role in understanding the underlying principles of how music is organized and composed, and analyzing the structure of music is a challenging task. 

Recently, Music Structure Analysis (MSA) has emerged as a subfield of Music Information Retrieval (MIR), aiming to develop computational methods to analyze the structure of music. This master thesis investigates the relationship between MSA, aural skills, and the underlying musical structure in musical composition. 

In recent years, the utilization of neural networks has allowed for the creation of low-dimensional musical embeddings that capture important musical features and enable the computation of semantic similarity between musical passages. By learning an embedding function $f(x) \in \mathbb{R}^d$ that maps a given passage $x$ to a vector in the real space $\mathbb{R}^d$, researchers can quantify the semantic similarity between two musical passages using the Euclidean distance, which is represented by the formula $D_{i,j} = ||f(x_i) - f(x_j)||^2$. Adequate embedding space is essential in ensuring that the generated output aligns with human expectations, especially in interactive applications where a human performer influences the machine's response. The success of such systems depends on the embedding space's ability to interpret music in a way that aligns with human perception.

The thesis explores the musical elements that constitute music\footnote{The term "music" used in this context refers specifically to Western tonal music tradition. It is assumed that the musical structures and elements discussed are based on this tradition and may not apply to other musical styles or cultures.} structure analysis, including melody, harmony, rhythm, and form. Aural skills involve perceptual and cognitive abilities that enable individuals to comprehend musical works through listening. The underlying musical structure refers to the systematic arrangement of musical elements that results in a coherent and expressive musical work. 

This thesis provides a comprehensive analysis of musical works from diverse styles and genres, utilizing computational techniques from MIR and incorporating a musicological perspective to contextualize the findings. 

This research aims to offer fresh insights into these components of musical composition and advance our understanding of their relationships.

\section{Motivation}

As a music lover and connoisseur, I am interested in gaining a deeper understanding of the fundamental building of musical composition, the various techniques used to create musical works, and capturing underlying relationships among them. I am deeply interested in learning how to retrieve embedded information within music. The study of music structure analysis provides a means to systematically analyze and understand musical pieces. It has traditionally been approached through music theory, formal musical notation, musical analysis techniques, and machine learning, which I plan to blend.

With the advent of artificial intelligence and machine learning, Music Information Retrieval (MIR) has expanded to encompass musically motivated neural network architectures. These networks are designed to incorporate musical concepts and perceptual approaches and use this knowledge to analyze and understand the structure of musical pieces in a way that goes beyond traditional music analysis techniques.

As a professional musician and music transcriber, I am interested in understanding how deep neural networks (DNNs) filter musical features. In other words, I am eager to learn how DNNs listen~\cite{7500246} to music.

By studying music structure analysis using musically motivated neural networks, we will gain an in-depth understanding of the latest advancements in the field and how these techniques are being used to revolutionize our understanding of music. Additionally, we will be able to contribute to developing new and innovative methods for musical analysis~\cite{Huang2019MusicTG}

\section{Objectives}

As a professional music transcriber, I aim to understand how deep neural networks (DNNs) filter musical features, listen to the NN layers from a musician's perspective and analyze the resulting data. I aim to approach the technology from a technical and musical standpoint to understand how DNNs process and extract musical information. By doing so, I aim to enhance the accuracy and efficiency of the process and contribute to developing AI technology for music analysis. By combining my expertise in music with my -not fully developed- knowledge of DNNs, I hope to bring a unique perspective to the field and contribute to the advancement of music analysis.

Music is a complex and multi-faceted phenomenon, and implementing a model that mimics human aural music skills is challenging. However, using the latest deep learning techniques, we aim to build a model that achieves good results for a specific task in the domain of music.

\section{Structure of the Report}

In this report, we have organized the information into several sections and sub-sections to provide a clear and comprehensive overview of the topic. The structure of the report is as follows:

\begin{itemize}
\item Chapter 1
  \begin{itemize}
  \item Purpose
  \item Methodology
  \item Scope
  \end{itemize}
\item Chapter 2
  \begin{itemize}
  \item Section 1
  \item Section 2
  \item Section 3
  \end{itemize}
\item Chapter 3
  \begin{itemize}
  \item Summary
  \item Recommendations
  \end{itemize}
\item References
  \begin{itemize}
  \item Sources Cited
  \end{itemize}
\end{itemize}

\newpage


