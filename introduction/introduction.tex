\normallinespacing

\chapter{Introduction}

Music Structure Analysis (MSA) is the field of study that aims to understand music's underlying organization and structure. This can include analyzing elements such as melody, harmony, rhythm, and timbre, as well as the relationships and interactions between these elements. MSA aims to identify patterns and regularities in music and develop computational models that can replicate and generate musical structures.

The fundamental concept behind MSA is the ability to divide a song into distinct, non-overlapping segments, assigning a label to each component that defines its type. This segmentation and labeling process aims to capture a human's perception or analysis of the song.

MSA is a multidisciplinary field that draws on techniques from music theory, signal processing, machine learning, and other areas. It can be applied to various musical styles and genres and has been used for tasks such as music transcription, generation, and similarity. 

\section{Topic overview}

Music Structure Analysis (MSA) has a long history, dating back to music theory and composition studies. In the 18th and 19th centuries, music theorists such as Rameau~\cite{christensen2004rameau} and Schenker~\cite{schenkerdocumentsonline} developed analytical frameworks for understanding the structure of music in terms of harmony, melody, and form.

\section{Motivation}

As a music lover and connoisseur, I am interested in gaining a deeper understanding of the fundamental building of musical composition and the various techniques used to create musical works~\cite{choi2017convolutional}. I am deeply interested in learning how to retrieve embedded and encoded information within the music. The study of music structure analysis provides a means to systematically analyze and understand musical pieces. It has traditionally been approached through music theory, formal musical notation, musical analysis techniques, and machine learning, which I plan to blend.

With the advent of artificial intelligence and machine learning, the field of music structure analysis has expanded to encompass the use of musically motivated neural network architectures. These networks are designed to incorporate musical concepts and perceptual approaches and use this knowledge to analyze and understand the structure of musical pieces in a way that goes beyond traditional music analysis techniques.

As a professional music transcriber, you are interested in understanding how deep neural networks (DNNs) filter musical features. In other words, I am eager to learn how DNNs listen~\cite{7500246} to music.

By studying music structure analysis using musically motivated neural networks, you will gain an in-depth understanding of the latest advancements in the field and how these techniques are being used to revolutionize our understanding of music. Additionally, we will be able to contribute to developing new and innovative methods for musical analysis~\cite{Huang2019MusicTG}

\section{Objectives}

In previous work, an overview of the current state of the art in MSA has been presented, including the most effective methods, principles, evaluation metrics, datasets, and algorithm performance~\cite{Nieto2020}~\cite{Chaki2021}.

As a professional music transcriber, I aim to understand how deep neural networks (DNNs) filter musical features, listen to the NN layers from a musician's perspective and analyze the resulting data. My goal is to approach the technology from both a technical and musical standpoint to understand how DNNs process and extract musical information. By doing so, you aim to enhance the accuracy and efficiency of the process and contribute to developing AI technology for music analysis. By combining my expertise in music with my -not fully developed- knowledge of DNNs, I hope to bring a unique perspective to the field and contribute to the advancement of music analysis.

\section{Structure of the Report}


\newpage


