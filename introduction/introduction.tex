\normallinespacing

\chapter{Introduction}

Music Structure Analysis (MSA) is a field of study that aims to understand music's underlying organization and structure. This can include analyzing elements such as melody, harmony, rhythm, and timbre, as well as the relationships and interactions between these elements. MSA aims to identify patterns and regularities in music and develop computational models that can replicate and generate musical structures.

The fundamental concept behind MSA is the ability to divide a song into distinct, non-overlapping segments, assigning a label to each segment that defines its type. This segmentation and labeling process aims to capture a human's perception or analysis of the song.

MSA is a multidisciplinary field that draws on techniques from music theory, signal processing, machine learning, and other fields. It can be applied to various musical styles and genres and has been used for tasks such as music transcription, music generation, and music similarity. 

\section{Motivation}

In previous work, an overview of the current state of the art in MSA has been presented, including the most effective methods, principles, evaluation metrics, datasets, and performance~\cite{Nieto2020}~\cite{Chaki2021}.

\section{Objectives}

\section{Structure of the Report}


\newpage


