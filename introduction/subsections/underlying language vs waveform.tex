At a high level, the underlying composition and the master recording represent different aspects of a musical work.

The underlying composition represents the musical and lyrical elements of the song in a fixed and notated form. It includes the melody, rhythm, and harmony of the song as well as the accompanying lyrics. The underlying composition is the intellectual property of the songwriters, composers, musicians, or publishers, who have exclusive rights to reproduce, distribute, and perform the work.

The master recording, on the other hand, is the tangible recording of the sound of the song. It captures the performance, production, and engineering aspects of the work. The master recording is the intellectual property of the performers, producers, or sound engineers, who have exclusive rights to reproduce, distribute, and prepare derivative works of the sound recording.

In essence, the underlying composition is a fixed and notated representation of the song's musical and lyrical elements, while the master recording is a tangible recording of the performance, production, and engineering aspects of the song. The underlying composition is owned by the songwriters, composers, musicians, or publishers, while the master recording is owned by the performers, producers, or sound engineers.