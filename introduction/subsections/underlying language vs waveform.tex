At a high level, the underlying composition and the master recording represent different aspects of a musical work.

The underlying composition represents the musical and lyrical elements of the song in a fixed, heuristic, and possibly notated form. It includes the melody, rhythm, harmony of the song, and accompanying lyrics, if any. In the lawsuit domain, the underlying composition is the intellectual property of the songwriters, composers, musicians, or publishers, who have exclusive rights to reproduce, distribute, and perform the work.

The master recording, on the other hand, is the tangible recording of the song's sound. It captures the performance, production, and engineering aspects of the work. The master recording is the intellectual property of the performers, producers, or sound engineers, who have exclusive rights to reproduce, distribute, and prepare derivative works of the sound recording.

The underlying composition is a fixed and notated representation of the song's musical and lyrical elements. At the same time, the master recording is a tangible recording of the song's performance, production, and engineering aspects. The underlying composition is owned by the songwriters, composers, musicians, or publishers, while the master recording is owned by the performers, producers, or sound engineers.