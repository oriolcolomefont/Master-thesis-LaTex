The underlying composition and master recording represent distinct and essential aspects of a musical work. The underlying composition consists of the musical and lyrical elements of the song, including the melody, harmony, rhythm, and lyrics, which are fixed and notated forms of the musical work. The composition represents the intellectual property of the songwriters, composers, musicians, or publishers, who have exclusive rights to reproduce, distribute, and perform the work.

The master recording, on the other hand, is a tangible recording that captures the sound of the work's performance, production, and engineering aspects. The master recording is owned by the performers, producers, or sound engineers, who have exclusive rights to reproduce, distribute, and prepare derivative works of the sound recording.

In the scientific and technical realm, the underlying composition is considered a fixed, heuristic representation of the musical work, whereas the master recording represents the sound recording of the performance. The composition is typically notated, meaning the musical elements are represented by symbols representing specific pitches, rhythms, and chords. This notation provides a way to represent the musical work in a standardized form that can be read and interpreted by musicians and musicologists alike.

The master recording, on the other hand, is a physical recording that captures the sound of the work's performance, production, and engineering aspects. This recording represents a tangible artifact that can be played back and studied in detail to understand the nuances of the performance, such as the tempo, dynamics, and tonal quality. In addition, the master recording can be manipulated through various audio production techniques to create derivative works, such as remixes or cover versions of the original song.

Overall, the underlying composition and master recording represent two essential aspects of a musical work, each with unique characteristics, properties, and ownership rights. The composition represents the fixed musical and lyrical elements of the work. It is owned by songwriters, composers, musicians, or publishers. At the same time, the master recording captures the performance and production aspects of the work and is owned by the performers, producers, or sound engineers.